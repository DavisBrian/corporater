\documentclass[]{book}
\usepackage{lmodern}
\usepackage{amssymb,amsmath}
\usepackage{ifxetex,ifluatex}
\usepackage{fixltx2e} % provides \textsubscript
\ifnum 0\ifxetex 1\fi\ifluatex 1\fi=0 % if pdftex
  \usepackage[T1]{fontenc}
  \usepackage[utf8]{inputenc}
\else % if luatex or xelatex
  \ifxetex
    \usepackage{mathspec}
  \else
    \usepackage{fontspec}
  \fi
  \defaultfontfeatures{Ligatures=TeX,Scale=MatchLowercase}
\fi
% use upquote if available, for straight quotes in verbatim environments
\IfFileExists{upquote.sty}{\usepackage{upquote}}{}
% use microtype if available
\IfFileExists{microtype.sty}{%
\usepackage{microtype}
\UseMicrotypeSet[protrusion]{basicmath} % disable protrusion for tt fonts
}{}
\usepackage[margin=1in]{geometry}
\usepackage{hyperref}
\hypersetup{unicode=true,
            pdftitle={Modern R in a Corporate Environment},
            pdfauthor={Brian Davis},
            pdfborder={0 0 0},
            breaklinks=true}
\urlstyle{same}  % don't use monospace font for urls
\usepackage{natbib}
\bibliographystyle{apalike}
\usepackage{color}
\usepackage{fancyvrb}
\newcommand{\VerbBar}{|}
\newcommand{\VERB}{\Verb[commandchars=\\\{\}]}
\DefineVerbatimEnvironment{Highlighting}{Verbatim}{commandchars=\\\{\}}
% Add ',fontsize=\small' for more characters per line
\usepackage{framed}
\definecolor{shadecolor}{RGB}{248,248,248}
\newenvironment{Shaded}{\begin{snugshade}}{\end{snugshade}}
\newcommand{\KeywordTok}[1]{\textcolor[rgb]{0.13,0.29,0.53}{\textbf{#1}}}
\newcommand{\DataTypeTok}[1]{\textcolor[rgb]{0.13,0.29,0.53}{#1}}
\newcommand{\DecValTok}[1]{\textcolor[rgb]{0.00,0.00,0.81}{#1}}
\newcommand{\BaseNTok}[1]{\textcolor[rgb]{0.00,0.00,0.81}{#1}}
\newcommand{\FloatTok}[1]{\textcolor[rgb]{0.00,0.00,0.81}{#1}}
\newcommand{\ConstantTok}[1]{\textcolor[rgb]{0.00,0.00,0.00}{#1}}
\newcommand{\CharTok}[1]{\textcolor[rgb]{0.31,0.60,0.02}{#1}}
\newcommand{\SpecialCharTok}[1]{\textcolor[rgb]{0.00,0.00,0.00}{#1}}
\newcommand{\StringTok}[1]{\textcolor[rgb]{0.31,0.60,0.02}{#1}}
\newcommand{\VerbatimStringTok}[1]{\textcolor[rgb]{0.31,0.60,0.02}{#1}}
\newcommand{\SpecialStringTok}[1]{\textcolor[rgb]{0.31,0.60,0.02}{#1}}
\newcommand{\ImportTok}[1]{#1}
\newcommand{\CommentTok}[1]{\textcolor[rgb]{0.56,0.35,0.01}{\textit{#1}}}
\newcommand{\DocumentationTok}[1]{\textcolor[rgb]{0.56,0.35,0.01}{\textbf{\textit{#1}}}}
\newcommand{\AnnotationTok}[1]{\textcolor[rgb]{0.56,0.35,0.01}{\textbf{\textit{#1}}}}
\newcommand{\CommentVarTok}[1]{\textcolor[rgb]{0.56,0.35,0.01}{\textbf{\textit{#1}}}}
\newcommand{\OtherTok}[1]{\textcolor[rgb]{0.56,0.35,0.01}{#1}}
\newcommand{\FunctionTok}[1]{\textcolor[rgb]{0.00,0.00,0.00}{#1}}
\newcommand{\VariableTok}[1]{\textcolor[rgb]{0.00,0.00,0.00}{#1}}
\newcommand{\ControlFlowTok}[1]{\textcolor[rgb]{0.13,0.29,0.53}{\textbf{#1}}}
\newcommand{\OperatorTok}[1]{\textcolor[rgb]{0.81,0.36,0.00}{\textbf{#1}}}
\newcommand{\BuiltInTok}[1]{#1}
\newcommand{\ExtensionTok}[1]{#1}
\newcommand{\PreprocessorTok}[1]{\textcolor[rgb]{0.56,0.35,0.01}{\textit{#1}}}
\newcommand{\AttributeTok}[1]{\textcolor[rgb]{0.77,0.63,0.00}{#1}}
\newcommand{\RegionMarkerTok}[1]{#1}
\newcommand{\InformationTok}[1]{\textcolor[rgb]{0.56,0.35,0.01}{\textbf{\textit{#1}}}}
\newcommand{\WarningTok}[1]{\textcolor[rgb]{0.56,0.35,0.01}{\textbf{\textit{#1}}}}
\newcommand{\AlertTok}[1]{\textcolor[rgb]{0.94,0.16,0.16}{#1}}
\newcommand{\ErrorTok}[1]{\textcolor[rgb]{0.64,0.00,0.00}{\textbf{#1}}}
\newcommand{\NormalTok}[1]{#1}
\usepackage{longtable,booktabs}
\usepackage{graphicx,grffile}
\makeatletter
\def\maxwidth{\ifdim\Gin@nat@width>\linewidth\linewidth\else\Gin@nat@width\fi}
\def\maxheight{\ifdim\Gin@nat@height>\textheight\textheight\else\Gin@nat@height\fi}
\makeatother
% Scale images if necessary, so that they will not overflow the page
% margins by default, and it is still possible to overwrite the defaults
% using explicit options in \includegraphics[width, height, ...]{}
\setkeys{Gin}{width=\maxwidth,height=\maxheight,keepaspectratio}
\IfFileExists{parskip.sty}{%
\usepackage{parskip}
}{% else
\setlength{\parindent}{0pt}
\setlength{\parskip}{6pt plus 2pt minus 1pt}
}
\setlength{\emergencystretch}{3em}  % prevent overfull lines
\providecommand{\tightlist}{%
  \setlength{\itemsep}{0pt}\setlength{\parskip}{0pt}}
\setcounter{secnumdepth}{5}
% Redefines (sub)paragraphs to behave more like sections
\ifx\paragraph\undefined\else
\let\oldparagraph\paragraph
\renewcommand{\paragraph}[1]{\oldparagraph{#1}\mbox{}}
\fi
\ifx\subparagraph\undefined\else
\let\oldsubparagraph\subparagraph
\renewcommand{\subparagraph}[1]{\oldsubparagraph{#1}\mbox{}}
\fi

%%% Use protect on footnotes to avoid problems with footnotes in titles
\let\rmarkdownfootnote\footnote%
\def\footnote{\protect\rmarkdownfootnote}

%%% Change title format to be more compact
\usepackage{titling}

% Create subtitle command for use in maketitle
\newcommand{\subtitle}[1]{
  \posttitle{
    \begin{center}\large#1\end{center}
    }
}

\setlength{\droptitle}{-2em}
  \title{Modern R in a Corporate Environment}
  \pretitle{\vspace{\droptitle}\centering\huge}
  \posttitle{\par}
\subtitle{Original materials developed for RADARS}
  \author{Brian Davis}
  \preauthor{\centering\large\emph}
  \postauthor{\par}
  \predate{\centering\large\emph}
  \postdate{\par}
  \date{2018-04-20}

\usepackage{booktabs}
\usepackage{amsthm}
\makeatletter
\def\thm@space@setup{%
  \thm@preskip=8pt plus 2pt minus 4pt
  \thm@postskip=\thm@preskip
}
\makeatother

\usepackage{amsthm}
\newtheorem{theorem}{Theorem}[chapter]
\newtheorem{lemma}{Lemma}[chapter]
\theoremstyle{definition}
\newtheorem{definition}{Definition}[chapter]
\newtheorem{corollary}{Corollary}[chapter]
\newtheorem{proposition}{Proposition}[chapter]
\theoremstyle{definition}
\newtheorem{example}{Example}[chapter]
\theoremstyle{definition}
\newtheorem{exercise}{Exercise}[chapter]
\theoremstyle{remark}
\newtheorem*{remark}{Remark}
\newtheorem*{solution}{Solution}
\begin{document}
\maketitle

{
\setcounter{tocdepth}{1}
\tableofcontents
}
\chapter*{About}\label{about}
\addcontentsline{toc}{chapter}{About}

\chapter{Introduction}\label{introduction}

\begin{quote}
Something that will make life easier in the long-run can be the most
difficult thing to do today. For coders, prioritising the long term may
involve an overhaul of current practice and the learning of a new skill.
\end{quote}

\section{Course Philosophy}\label{course-philosophy}

\begin{quote}
``The best programs are written so that computing machines can perform
them quickly and so that human beings can understand them clearly. A
programmer is ideally an essayist who works with traditional aesthetic
and literary forms as well as mathematical concepts, to communicate the
way that an algorithm works and to convince a reader that the results
will be correct.'' Donald Knuth
\end{quote}

\subsection{Reproducible Research}\label{reproducible-research}

Reproducible research is the idea that data analyses, and more
generally, scientific claims, are published with their data and software
code so that others may verify the findings and build upon them. There
are two basic reasons to be concerned about making your research
reproducible. The first is \emph{to show evidence of the correctness of
your results}. The second reason to aspire to reproducibility is
\emph{to enable others to make use of our methods and results}.

Modern challenges of reproducibility in research, particularly
computational reproducibility, have produced a lot of discussion in
papers, blogs and videos, some of which are listed
\href{http://ropensci.github.io/reproducibility-guide/sections/references/}{here}
and \href{https://reproducibleresearch.net/}{here}.

\begin{quote}
Conclusions in experimental psychology often are the result of null
hypothesis significance testing. Unfortunately, there is evidence ((from
eight major psychology journals published between 1985 and 2013) that
roughly half of all published empirical psychology articles contain at
least one inconsistent p-value, and around one in eight articles contain
a grossly inconsistent p-value that makes a non-significant result seem
significant, or vice versa.
\href{https://mbnuijten.com/statcheck/}{statscheck} and
\href{http://blog.revolutionanalytics.com/2016/10/statcheck.html}{here}
\end{quote}

\begin{quote}
``A key component of scientific communication is sufficient information
for other researchers in the field to reproduce published findings. For
computational and data-enabled research, this has often been interpreted
to mean making available the raw data from which results were generated,
the computer code that generated the findings, and any additional
information needed such as workflows and input parameters. Many journals
are revising author guidelines to include data and code availability. We
chose a random sample of 204 scientific papers published in the journal
\textbf{Science} after the implementation of their policy in February
2011. We found that were able to reproduce the findings for 26\%.''
\href{http://www.pnas.org/content/115/11/2584}{Proceedings of the
National Academy of Sciences of the United States of America}
\end{quote}

\begin{quote}
``Starting September 1 2016, JASA ACS will require code and data as a
minimum standard for reproducibility of statistical scientific
research.''
\href{https://magazine.amstat.org/blog/2016/07/01/jasa-reproducible16/}{JASA}
\end{quote}

\subsection{FDA Validation}\label{fda-validation}

\begin{quote}
``Establishing documented evidence which provides a high degree of
assurance that a specific process will consistently produce a product
meeting its predetermined specifications and quality attributes.''
-Validation as defined by the FDA in \textbf{Validation of Systems for
21 CFR Part 11 Compliance}
\end{quote}

\subsection{The SAS Myth}\label{the-sas-myth}

Contray to what we hear the FDA does not require SAS to be used
\emph{EVER}. There are instances that you have to deliver data in XPORT
format though which is open and implemented in many programming
languages.

\begin{quote}
``FDA does not require use of any specific software for statistical
analyses, and statistical software is not explicitly discussed in Title
21 of the Code of Federal Regulations {[}e.g., in 21CFR part 11{]}.
However, the software package(s) used for statistical analyses should be
fully documented in the submission, including version and build
identification. As noted in the FDA guidance, E9 Statistical Principles
for Clinical Trials''
\href{https://www.fda.gov/downloads/forindustry/datastandards/studydatastandards/ucm587506.pdf}{FDA
Statistical Software Clarifying Statement}
\end{quote}

Good \href{http://blog.revolutionanalytics.com/2017/06/r-fda.html}{write
up} with links to several FDA talks on the subject.

\section{Prerequisites}\label{prerequisites}

\begin{itemize}
\tightlist
\item
  We will assume you have minimal experience and knowledge of R
\item
  IT should have installed:

  \begin{itemize}
  \tightlist
  \item
    \href{https://cran.r-project.org/}{R} version 3.5
  \item
    \href{https://www.rstudio.com/products/rstudio/download/\#download}{RStudio}
    version 1.1
  \item
    \href{https://miktex.org/}{MiTeX}
  \item
    \href{https://cran.r-project.org/bin/windows/Rtools/}{RTools}
    version 3.4
  \end{itemize}
\item
  We will install other dependencies throughout the course.
\end{itemize}

\section{Content}\label{content}

It is impossible to become an expert in R in only one course even a
multi-week one. Yet, this course aims at giving a wide understanding on
many aspects of R as used in a corporate / production environment. It
will roughly be based on \href{http://r4ds.had.co.nz}{R for Data
Science}. While this is an \emph{excellent} resource it does not cover
much of what we will need on a routine basis. Some external resources
will be referred to in this book for you to be able to deepen what you
would have learned in this course.

This is your course so if you feel we need to hit an area deeper, or add
content based on a current need, let me know an we will work to adjust
it.

The \textbf{rough} topic list of the course:

\begin{enumerate}
\def\labelenumi{\arabic{enumi}.}
\tightlist
\item
  Good programming practices
\item
  Basics of R Programming
\item
  Importing Data
\item
  Tiyding Data
\item
  Visualizing Data
\item
  Functions
\item
  Strings
\item
  Dates and Time
\item
  Communicating Results
\end{enumerate}

Making Code Production Ready:

\begin{enumerate}
\def\labelenumi{\arabic{enumi}.}
\setcounter{enumi}{9}
\tightlist
\item
  Functions (part II)
\item
  Assertions
\item
  Unit tests
\item
  Documentation
\item
  Communicating Results (part II)
\end{enumerate}

\section{Structure}\label{structure}

My current thoughts are to meet an hour a week and discuss a topic. We
will not be going strictly through the R4DS, but will use it as our
foundation into the topic at hand. Then give an assignment due for the
next week which we go over the solutions. We will incorporate these
assignments into a RADARS R package(s?) so we will have a collection of
usefull reusable code for the future.

Open to other ideas as we go along. I'm going to try to keep the
assignments related to our current work (maybe working through Site
Investigator and/or Subscriber Reports) so we can work on the class
during work hours.

\chapter{Good practices}\label{good-practices}

\begin{quote}
``Programs must be written for people to read, and only incidentally for
machines to execute.'' Harold Abelson
\end{quote}

\begin{quote}
``Programming is the art of telling another human being what one wants
the computer to do.'' Donald Knuth
\end{quote}

\begin{quote}
``Let us change our traditional attitude to the construction of
programs. Instead of imagining that our main task is to instruct a
computer what to do, let us concentrate rather on explaining to human
beings what we want a computer to do.'' Donald Knuth
\end{quote}

\begin{quote}
``When you write a program, think of it primarily as a work of
literature. You're trying to write something that human beings are going
to read. Don't think of it primarily as something a computer is going to
follow. The more effective you are at making your program readable, the
more effective it's going to be: You'll understand it today, you'll
understand it next week, and your successors who are going to maintain
and modify it will understand it.''
\end{quote}

\section{Coding style}\label{coding-style}

Good coding style is like correct punctuation: you can manage without
it, butitsuremakesthingseasiertoread. When I answer questions; first, I
read the title of the question to see if I can answer the question,
secondly, I check the coding style of the question and if the code is
too difficult to read, I just move on. Please make your code readable by
following e.g. \href{http://style.tidyverse.org/}{this coding style}
(most examples below come from this guide).

\subsection{Comments}\label{comments}

In code, use comments to explain the ``why'' not the ``what'' or
``how''. Each line of a comment should begin with the comment symbol and
a single space: \texttt{\#}.

\subsection{Naming}\label{naming}

\begin{quote}
There are only two hard things in Computer Science: cache invalidation
and naming things. -- Phil Karlton
\end{quote}

Names are not limited to 8 characters as in some other languages. Be
smart with your naming; be descriptive yet concise. Think about how your
names will show up in autocomplete.

Throughout the course we will point out some standard naming conventions
that are used in R (and other languages). (Ex. \texttt{i} and \texttt{j}
as row and column indicies)

\begin{Shaded}
\begin{Highlighting}[]
\CommentTok{# Good}
\NormalTok{average_height <-}\StringTok{ }\KeywordTok{mean}\NormalTok{((feet }\OperatorTok{/}\StringTok{ }\DecValTok{12}\NormalTok{) }\OperatorTok{+}\StringTok{ }\NormalTok{inches)}
\KeywordTok{plot}\NormalTok{(mtcars}\OperatorTok{$}\NormalTok{disp, mtcars}\OperatorTok{$}\NormalTok{mpg)}

\CommentTok{# Bad}
\NormalTok{ah<-}\KeywordTok{mean}\NormalTok{(x}\OperatorTok{/}\DecValTok{12}\OperatorTok{+}\NormalTok{y)}
\KeywordTok{plot}\NormalTok{(mtcars[, }\DecValTok{3}\NormalTok{], mtcars[, }\DecValTok{1}\NormalTok{])}
\end{Highlighting}
\end{Shaded}

\subsection{Structure}\label{structure-1}

Use commented lines of \texttt{-} to create a code outline.

\subsection{Spacing}\label{spacing}

Put a space before and after \texttt{=} when naming arguments in
function calls. Most infix operators (\texttt{==}, \texttt{+},
\texttt{-}, \texttt{\textless{}-}, etc.) are also surrounded by spaces,
except those with relatively high precedence: \texttt{\^{}}, \texttt{:},
\texttt{::}, and \texttt{:::}. Always put a space after a comma, and
never before (just like in regular English).

\begin{Shaded}
\begin{Highlighting}[]
\CommentTok{# Good}
\NormalTok{average <-}\StringTok{ }\KeywordTok{mean}\NormalTok{((feet }\OperatorTok{/}\StringTok{ }\DecValTok{12}\NormalTok{) }\OperatorTok{+}\StringTok{ }\NormalTok{inches, }\DataTypeTok{na.rm =} \OtherTok{TRUE}\NormalTok{)}
\KeywordTok{sqrt}\NormalTok{(x}\OperatorTok{^}\DecValTok{2} \OperatorTok{+}\StringTok{ }\NormalTok{y}\OperatorTok{^}\DecValTok{2}\NormalTok{)}
\NormalTok{x <-}\StringTok{ }\DecValTok{1}\OperatorTok{:}\DecValTok{10}
\NormalTok{base}\OperatorTok{::}\NormalTok{sum}

\CommentTok{# Bad}
\NormalTok{average<-}\KeywordTok{mean}\NormalTok{(feet}\OperatorTok{/}\DecValTok{12}\OperatorTok{+}\NormalTok{inches,}\DataTypeTok{na.rm=}\OtherTok{TRUE}\NormalTok{)}
\KeywordTok{sqrt}\NormalTok{(x }\OperatorTok{^}\StringTok{ }\DecValTok{2} \OperatorTok{+}\StringTok{ }\NormalTok{y }\OperatorTok{^}\StringTok{ }\DecValTok{2}\NormalTok{)}
\NormalTok{x <-}\StringTok{ }\DecValTok{1} \OperatorTok{:}\StringTok{ }\DecValTok{10}
\NormalTok{base }\OperatorTok{::}\StringTok{ }\NormalTok{sum}
\end{Highlighting}
\end{Shaded}

\subsection{Indenting}\label{indenting}

Curly braces, \texttt{\{\}}, define the the most important hierarchy of
R code. To make this hierarchy easy to see, always indent the code
inside \texttt{\{\}} by two spaces.

\begin{Shaded}
\begin{Highlighting}[]
\CommentTok{# Good}
\ControlFlowTok{if}\NormalTok{ (y }\OperatorTok{<}\StringTok{ }\DecValTok{0} \OperatorTok{&&}\StringTok{ }\NormalTok{debug) \{}
  \KeywordTok{message}\NormalTok{(}\StringTok{"y is negative"}\NormalTok{)}
\NormalTok{\}}

\ControlFlowTok{if}\NormalTok{ (y }\OperatorTok{==}\StringTok{ }\DecValTok{0}\NormalTok{) \{}
  \ControlFlowTok{if}\NormalTok{ (x }\OperatorTok{>}\StringTok{ }\DecValTok{0}\NormalTok{) \{}
    \KeywordTok{log}\NormalTok{(x)}
\NormalTok{  \} }\ControlFlowTok{else}\NormalTok{ \{}
    \KeywordTok{message}\NormalTok{(}\StringTok{"x is negative or zero"}\NormalTok{)}
\NormalTok{  \}}
\NormalTok{\} }\ControlFlowTok{else}\NormalTok{ \{}
\NormalTok{  y }\OperatorTok{^}\StringTok{ }\NormalTok{x}
\NormalTok{\}}

\CommentTok{# Bad}
\ControlFlowTok{if}\NormalTok{ (y }\OperatorTok{<}\StringTok{ }\DecValTok{0} \OperatorTok{&&}\StringTok{ }\NormalTok{debug)}
\KeywordTok{message}\NormalTok{(}\StringTok{"Y is negative"}\NormalTok{)}

\ControlFlowTok{if}\NormalTok{ (y }\OperatorTok{==}\StringTok{ }\DecValTok{0}\NormalTok{)}
\NormalTok{\{}
    \ControlFlowTok{if}\NormalTok{ (x }\OperatorTok{>}\StringTok{ }\DecValTok{0}\NormalTok{) \{}
      \KeywordTok{log}\NormalTok{(x)}
\NormalTok{    \} }\ControlFlowTok{else}\NormalTok{ \{}
  \KeywordTok{message}\NormalTok{(}\StringTok{"x is negative or zero"}\NormalTok{)}
\NormalTok{    \}}
\NormalTok{\} }\ControlFlowTok{else}\NormalTok{ \{ y }\OperatorTok{^}\StringTok{ }\NormalTok{x \}}
\end{Highlighting}
\end{Shaded}

\subsection{Long lines}\label{long-lines}

Strive to limit your code to 80 characters per line. This fits
comfortably on a printed page with a reasonably sized font. If you find
yourself running out of room, this is a good indication that you should
encapsulate some of the work into a separate function.

If a function call is too long to fit on a single line, use one line
each for the function name, each argument, and the closing \texttt{)}.
This makes the code easier to read and to change later.

\begin{Shaded}
\begin{Highlighting}[]
\CommentTok{# Good}
\KeywordTok{do_something_very_complicated}\NormalTok{(}
  \DataTypeTok{something =} \StringTok{"that"}\NormalTok{,}
  \DataTypeTok{requires  =}\NormalTok{ many,}
  \DataTypeTok{arguments =} \StringTok{"some of which may be long"}
\NormalTok{)}

\CommentTok{# Bad}
\KeywordTok{do_something_very_complicated}\NormalTok{(}\StringTok{"that"}\NormalTok{, requires, many, arguments,}
                              \StringTok{"some of which may be long"}
\end{Highlighting}
\end{Shaded}

\subsection{Other}\label{other}

\begin{itemize}
\tightlist
\item
  Use \texttt{\textless{}-}, not \texttt{=}, for assignment. Keep
  \texttt{=} for parameters.
\end{itemize}

\begin{Shaded}
\begin{Highlighting}[]
\CommentTok{# Good}
\NormalTok{x <-}\StringTok{ }\DecValTok{5}
\KeywordTok{system.time}\NormalTok{(}
\NormalTok{  x <-}\StringTok{ }\KeywordTok{rnorm}\NormalTok{(}\FloatTok{1e6}\NormalTok{)}
\NormalTok{)}

\CommentTok{# Bad}
\NormalTok{x =}\StringTok{ }\DecValTok{5}
\KeywordTok{system.time}\NormalTok{(}
  \DataTypeTok{x =} \KeywordTok{rnorm}\NormalTok{(}\FloatTok{1e6}\NormalTok{)}
\NormalTok{)}
\end{Highlighting}
\end{Shaded}

\begin{itemize}
\item
  Don't put \texttt{;} at the end of a line, and don't use \texttt{;} to
  put multiple commands on one line.
\item
  Only use \texttt{return()} for early returns. Otherwise rely on R to
  return the result of the last evaluated expression.
\end{itemize}

\begin{Shaded}
\begin{Highlighting}[]
\CommentTok{# Good}
\NormalTok{add_two <-}\StringTok{ }\ControlFlowTok{function}\NormalTok{(x, y) \{}
\NormalTok{  x }\OperatorTok{+}\StringTok{ }\NormalTok{y}
\NormalTok{\}}

\CommentTok{# Bad}
\NormalTok{add_two <-}\StringTok{ }\ControlFlowTok{function}\NormalTok{(x, y) \{}
  \KeywordTok{return}\NormalTok{(x }\OperatorTok{+}\StringTok{ }\NormalTok{y)}
\NormalTok{\}}
\end{Highlighting}
\end{Shaded}

\begin{itemize}
\tightlist
\item
  Use \texttt{"}, not \texttt{\textquotesingle{}}, for quoting text. The
  only exception is when the text already contains double quotes and no
  single quotes.
\end{itemize}

\begin{Shaded}
\begin{Highlighting}[]
\CommentTok{# Good}
\StringTok{"Text"}
\StringTok{'Text with "quotes"'}
\StringTok{'<a href="http://style.tidyverse.org">A link</a>'}

\CommentTok{# Bad}
\StringTok{'Text'}
\StringTok{'Text with "double" and }\CharTok{\textbackslash{}'}\StringTok{single}\CharTok{\textbackslash{}'}\StringTok{ quotes'}
\end{Highlighting}
\end{Shaded}

\section{Coding practices}\label{coding-practices}

\subsection{Variables}\label{variables}

Create variables for values that are likely to change.

\subsection{\texorpdfstring{\emph{Rule of
3}}{Rule of 3}}\label{rule-of-3}

Try not to copy code, or copy then modify the code, more than twice.

\begin{itemize}
\tightlist
\item
  If a change requires you to search/replace 3 or more times make a
  variable.
\item
  If you copy a code chunk 3 or more times \emph{make a function}
\item
  If you copy a function 3 or more times \emph{make your function more
  generic}
\item
  If you copy a function 3 or more times into a project \emph{make a
  package}
\item
  If 3 or more people will use the function \emph{make a package}
\item
  If 3 or more projects will use the function \emph{make a package}
\end{itemize}

Same thing goes for lookup tables and such. The key thing to think about
is; if something changes how many touch points will there be? If it is 3
or more places it is time to abstract this code a bit.

\subsection{Path names}\label{path-names}

It is better to use relative path names instead of hard coded ones. If
you must read from (or write to) paths that are not in your project
directory structure create a file name variable at the highest level you
can (\emph{always end with the \texttt{/}}) and then use relative
paths.\\
\textbf{DO NOT EVER USE \texttt{setwd()}}

\begin{Shaded}
\begin{Highlighting}[]
\CommentTok{# Good}
\NormalTok{raw_data <-}\StringTok{ }\KeywordTok{read.csv}\NormalTok{(}\StringTok{"./data/mydatafile.csv"}\NormalTok{) }

\NormalTok{input_file <-}\StringTok{ "./data/mydatafile.csv"}
\NormalTok{raw_data <-}\StringTok{ }\KeywordTok{read.csv}\NormalTok{(input_file)  }

\NormalTok{input_path <-}\StringTok{ "C:/Path/To/Some/other/project/directory/"}
\NormalTok{input_file <-}\StringTok{ }\KeywordTok{paste0}\NormalTok{(input_path, }\StringTok{"data/mydatafile.csv"}\NormalTok{)}
\NormalTok{raw_data <-}\StringTok{ }\KeywordTok{read.csv}\NormalTok{(input_file)}

\CommentTok{# Bad}
\KeywordTok{setwd}\NormalTok{(}\StringTok{"C:/Path/To/Some/other/project/directory/data/"}\NormalTok{)}
\NormalTok{raw_data <-}\StringTok{ }\KeywordTok{read.csv}\NormalTok{(}\StringTok{"mydatafile.csv"}\NormalTok{)}
\KeywordTok{setwd}\NormalTok{(}\StringTok{"C:/Path/back/to/my/project/"}\NormalTok{)}
\end{Highlighting}
\end{Shaded}

\section{RStudio}\label{rstudio}

Download the latest version of
\href{https://www.rstudio.com/products/rstudio/download/\#download}{RStudio}
(\textgreater{} 1.1) and use it!

Learn more about new features of RStudio v1.1
\href{https://www.rstudio.com/resources/videos/rstudio-1-1-new-features/}{there}.

RStudio features:

\begin{itemize}
\item
  everything you can expect from a good IDE
\item
  keyboard shortcuts I use frequently

  \begin{enumerate}
  \def\labelenumi{\arabic{enumi}.}
  \tightlist
  \item
    \emph{Ctrl + Space} (auto-completion, better than \emph{Tab})
  \item
    \emph{Ctrl + Up} (command history \& search)
  \item
    \emph{Ctrl + Enter} (execute line of code)
  \item
    \emph{Ctrl + Shift + A} (reformat code)
  \item
    \emph{Ctrl + Shift + C} (comment/uncomment selected lines)
  \item
    \emph{Ctrl + Shift + /} (reflow comments)
  \item
    \emph{Ctrl + Shift + O} (View code outline)
  \item
    \emph{Ctrl + Shift + B} (build package, website or book)
  \item
    \emph{Ctrl + Shift + M} (pipe)
  \item
    \emph{Alt + Shift + K} to see all shortcuts\ldots{}
  \end{enumerate}
\item
  Panels (everything is integrated, including \textbf{Git} and a
  terminal)
\item
  Interactive data importation from files and connections (see
  \href{https://www.rstudio.com/resources/webinars/importing-data-into-r/}{this
  webinar})
\item
  Use
  \href{https://support.rstudio.com/hc/en-us/articles/205753617-Code-Diagnostics}{code
  diagnostics}:
\item
  \textbf{R Projects}:

  \begin{itemize}
  \tightlist
  \item
    \textbf{Meaningful structure} in one folder
  \item
    The working directory automatically switches to the project's folder
  \item
    File tab displays the associated files and folders in the project
  \item
    History of R commands and open files
  \item
    Any settings associated with the project, such as Git settings, are
    loaded. Note that a \emph{set-up.R} or even a \emph{.Rprofile} file
    in the project's root directory enable project-specific settings to
    be loaded each time people work on the project.
  \end{itemize}
\end{itemize}

The only two things that make @JennyBryan 😤😠🤯. Instead use projects +
here::here() \#rstats pic.twitter.com/GwxnHePL4n

--- Hadley Wickham (@hadleywickham) December 11 2017

Read more at
\url{https://www.tidyverse.org/articles/2017/12/workflow-vs-script/} and
also see chapter
\href{https://bookdown.org/csgillespie/efficientR/set-up.html}{\emph{Efficient
set-up}} of book \emph{Efficient R programming}.

\section{Getting help}\label{getting-help}

\subsection{Help yourself, learn how to
debug}\label{help-yourself-learn-how-to-debug}

A basic solution is to print everything, but it usually does not work
well on complex problems. A convenient solution to see all the
variables' states in your code is to place some \texttt{browser()}
anywhere you want to check the variables' states.

Learn more with
\href{https://bookdown.org/rdpeng/rprogdatascience/debugging.html}{this
book chapter},
\href{http://adv-r.had.co.nz/Exceptions-Debugging.html}{this other book
chapter},
\href{https://www.rstudio.com/resources/videos/debugging-techniques-in-rstudio/}{this
webinar} and
\href{https://support.rstudio.com/hc/en-us/articles/205612627-Debugging-with-RStudio}{this
RStudio article}.

\subsection{External help}\label{external-help}

Can't remember useful functions? Use
\href{https://www.rstudio.com/resources/cheatsheets/}{cheat sheets}.

You can search for specific R stuff on \url{https://rseek.org/}. You
should also read documentations carefully. If you're using a package,
search for vignettes and a GitHub repository.

You can also use \href{https://stackoverflow.com/}{Stack Overflow}. The
most common use of Stack Overflow is when you have an error or a
question, you google it, and most of the times the first links are Q/A
on Stack Overflow.

You can ask questions on Stack Overflow (using the tag \texttt{r}). You
need to
\href{https://stackoverflow.com/questions/5963269/how-to-make-a-great-r-reproducible-example}{make
a great R reproducible example} if you want your question to be
answered. Most of the times, while making this reproducible example, you
will find the answer to your problem.

If you're confident enough in your R skills, you can go to the next step
and
\href{https://stackoverflow.com/unanswered/tagged/r?tab=newest}{answer
questions on Stack Overflow}. It's a good way to increase your skills,
or just to
\href{https://privefl.github.io/blog/one-month-as-a-procrastinator-on-stack-overflow/}{procrastinate
while writing a scientific manuscript}.

\section{Keeping up to date}\label{keeping-up-to-date}

With over 10,000 packages on CRAN it is hard to keep up with the
constantly changing landscape.
\href{https://www.r-bloggers.com/}{R-Bloggers} is an R forcused blog
aggregator with dozens of posts per day. Checkit out.

Join the \href{https://www.r-project.org/mail.html}{R-help} mailing
list. Sign up to get the daily digest and scan it for questions that
interest you.

\section{Assignement}\label{assignement}

\begin{enumerate}
\def\labelenumi{\arabic{enumi}.}
\tightlist
\item
  See these Rstudio
  \href{https://rviews.rstudio.com/categories/tips-and-tricks/}{Tips \&
  Tricks} or \href{https://twitter.com/rstudiotips}{these} and find one
  that looks interesting and \textbf{practice} it all week.
\item
  Create an R Project for this class.
\item
  Create the following directories in your project (tip sheet?)

  \begin{itemize}
  \tightlist
  \item
    Bonus points if you can do it from R and not RStudio or Windows
    Explorer
  \item
    Double Bonus points if you can make it a function.
  \end{itemize}
\item
  Read Chapters 1-3 of the
  \href{http://style.tidyverse.org/index.html}{Tidyverse Style Guide}
\item
  Copy one of your R scripts into your R directory. (Bonus points if you
  can do it from R and not RStudio or Windows Explorer)
\item
  Apply the style guide to your code.\\
\item
  Apply the ``Rule of 3''

  \begin{itemize}
  \tightlist
  \item
    Create variables as needed
  \item
    Identify code that is used 3 or more times to make functions
  \item
    Identify code that would be useful in 3 or more projects to
    integrate into a package.
  \end{itemize}
\item
  Read how to
  \href{https://stackoverflow.com/questions/5963269/how-to-make-a-great-r-reproducible-example}{make
  a great R reproducible example}
\end{enumerate}

\chapter{R Base Basics}\label{r-base-basics}

See this \href{http://adv-r.had.co.nz/Vocabulary.html}{vocabulary list}
for a good starting point on the basics functions in base R and some
important libraries.

\href{https://privefl.github.io/advr38book/r-programming.html}{advr38book}

In R there three basic constructs; objects, functions, and environments.

The three most important functions in R \texttt{?}, \texttt{??}, and
\texttt{str}.

\section{Nameing Rules}\label{nameing-rules}

R has strict rules about what constitutes a valid name. A
\textbf{syntactic} name must consist of letters\footnote{Surprisingly,
  what constitutes a letter is determined by your current locale. That
  means that the syntax of R code actually differs from computer to
  computer, and it's possible for a file that works on one computer to
  not even parse on another!}, digits, \texttt{.} and \texttt{\_}, and
can't begin with \texttt{\_}. Additionally, it can not be one of a list
of \textbf{reserved words} like \texttt{TRUE}, \texttt{NULL},
\texttt{if}, and \texttt{function} (see the complete list in
\texttt{?Reserved}). Names that don't follow these rules are called
\textbf{non-syntactic} names, and if you try to use them, you'll get an
error:

\begin{Shaded}
\begin{Highlighting}[]
\NormalTok{_abc <-}\StringTok{ }\DecValTok{1}
\CommentTok{#> Error: unexpected input in "_"}

\ControlFlowTok{if}\NormalTok{ <-}\StringTok{ }\DecValTok{10}
\CommentTok{#> Error: unexpected assignment in "if <-"}
\end{Highlighting}
\end{Shaded}

\section{Vectors}\label{vectors}

The most common data structure in R is the vector. R's vectors can be
organised by their dimensionality (1d, 2d, or nd) and whether they're
homogeneous or heterogeneous. This gives rise to the five data types
most often used in data analysis:

\begin{longtable}[]{@{}lll@{}}
\toprule
& Homogeneous & Heterogeneous\tabularnewline
\midrule
\endhead
1d & Atomic vector & List\tabularnewline
2d & Matrix & Data frame\tabularnewline
nd & Array &\tabularnewline
\bottomrule
\end{longtable}

Given an object, the best way to understand what data structures it is
composed of is to use \texttt{str()}. \texttt{str()} is short for
structure and it gives a compact, human readable description of any R
data structure.

Vectors have three common properties:

\begin{itemize}
\tightlist
\item
  Type, \texttt{typeof()}, what it is.
\item
  Length, \texttt{length()}, how many elements it contains.
\item
  Attributes, \texttt{attributes()}, additional arbitrary metadata.
\end{itemize}

They differ in the types of their elements: all elements of an atomic
vector must be the same type, whereas the elements of a list can have
different types.

NOTE: \texttt{is.vector()} does not test if an object is a vector.
Instead it returns TRUE only if the object is a vector with no
attributes apart from names. Use
\texttt{is.atomic(x)\ \textbar{}\textbar{}\ is.list(x)} to test if an
object is actually a vector.

\subsection{Atomic Vectors}\label{atomic-vectors}

There are many ``atomic'' types of data: \texttt{logical},
\texttt{integer}, \texttt{double} and \texttt{character} (in this order,
see below). There are also \texttt{raw} and \texttt{complex} but they
are rarely used.

You can't mix types in an atomic vector (you can in a list). Coercion
will automatically occur if you mix types:

\begin{Shaded}
\begin{Highlighting}[]
\NormalTok{(a <-}\StringTok{ }\OtherTok{FALSE}\NormalTok{)}
\end{Highlighting}
\end{Shaded}

\begin{verbatim}
#> [1] FALSE
\end{verbatim}

\begin{Shaded}
\begin{Highlighting}[]
\KeywordTok{typeof}\NormalTok{(a)}
\end{Highlighting}
\end{Shaded}

\begin{verbatim}
#> [1] "logical"
\end{verbatim}

\begin{Shaded}
\begin{Highlighting}[]
\NormalTok{(b <-}\StringTok{ }\DecValTok{1}\OperatorTok{:}\DecValTok{10}\NormalTok{)}
\end{Highlighting}
\end{Shaded}

\begin{verbatim}
#>  [1]  1  2  3  4  5  6  7  8  9 10
\end{verbatim}

\begin{Shaded}
\begin{Highlighting}[]
\KeywordTok{typeof}\NormalTok{(b)}
\end{Highlighting}
\end{Shaded}

\begin{verbatim}
#> [1] "integer"
\end{verbatim}

\begin{Shaded}
\begin{Highlighting}[]
\KeywordTok{c}\NormalTok{(a, b)         ## FALSE is coerced to integer 0}
\end{Highlighting}
\end{Shaded}

\begin{verbatim}
#>  [1]  0  1  2  3  4  5  6  7  8  9 10
\end{verbatim}

\begin{Shaded}
\begin{Highlighting}[]
\NormalTok{(c <-}\StringTok{ }\FloatTok{10.5}\NormalTok{)}
\end{Highlighting}
\end{Shaded}

\begin{verbatim}
#> [1] 10.5
\end{verbatim}

\begin{Shaded}
\begin{Highlighting}[]
\KeywordTok{typeof}\NormalTok{(c)}
\end{Highlighting}
\end{Shaded}

\begin{verbatim}
#> [1] "double"
\end{verbatim}

\begin{Shaded}
\begin{Highlighting}[]
\NormalTok{(d <-}\StringTok{ }\KeywordTok{c}\NormalTok{(b, c))  ## coerced to double}
\end{Highlighting}
\end{Shaded}

\begin{verbatim}
#>  [1]  1.0  2.0  3.0  4.0  5.0  6.0  7.0  8.0  9.0 10.0 10.5
\end{verbatim}

\begin{Shaded}
\begin{Highlighting}[]
\KeywordTok{c}\NormalTok{(d, }\StringTok{"a"}\NormalTok{)       ## coerced to character}
\end{Highlighting}
\end{Shaded}

\begin{verbatim}
#>  [1] "1"    "2"    "3"    "4"    "5"    "6"    "7"    "8"    "9"    "10"   "10.5" "a"
\end{verbatim}

\begin{Shaded}
\begin{Highlighting}[]
\KeywordTok{c}\NormalTok{(}\KeywordTok{list}\NormalTok{(}\DecValTok{1}\NormalTok{), }\StringTok{"a"}\NormalTok{)}
\end{Highlighting}
\end{Shaded}

\begin{verbatim}
#> [[1]]
#> [1] 1
#> 
#> [[2]]
#> [1] "a"
\end{verbatim}

\begin{Shaded}
\begin{Highlighting}[]
\DecValTok{50} \OperatorTok{<}\StringTok{ "7"}
\end{Highlighting}
\end{Shaded}

\begin{verbatim}
#> [1] TRUE
\end{verbatim}

You can force coercion with \texttt{as.logical}, \texttt{as.integer},
\texttt{as.double}, \texttt{as.numeric}, and \texttt{as.character}. Most
of the time the coercion rules are straight forward, but not always.

\begin{Shaded}
\begin{Highlighting}[]
\NormalTok{x <-}\StringTok{ }\KeywordTok{c}\NormalTok{(}\OtherTok{TRUE}\NormalTok{, }\OtherTok{FALSE}\NormalTok{)}
\KeywordTok{typeof}\NormalTok{(x)}
\end{Highlighting}
\end{Shaded}

\begin{verbatim}
#> [1] "logical"
\end{verbatim}

\begin{Shaded}
\begin{Highlighting}[]
\KeywordTok{as.integer}\NormalTok{(x)}
\end{Highlighting}
\end{Shaded}

\begin{verbatim}
#> [1] 1 0
\end{verbatim}

\begin{Shaded}
\begin{Highlighting}[]
\KeywordTok{as.numeric}\NormalTok{(x)}
\end{Highlighting}
\end{Shaded}

\begin{verbatim}
#> [1] 1 0
\end{verbatim}

\begin{Shaded}
\begin{Highlighting}[]
\KeywordTok{as.character}\NormalTok{(x)}
\end{Highlighting}
\end{Shaded}

\begin{verbatim}
#> [1] "TRUE"  "FALSE"
\end{verbatim}

However, coercion is not associative.

\begin{Shaded}
\begin{Highlighting}[]
\NormalTok{x <-}\StringTok{ }\KeywordTok{c}\NormalTok{(}\OtherTok{TRUE}\NormalTok{, }\OtherTok{FALSE}\NormalTok{)}

\NormalTok{x2 <-}\StringTok{ }\KeywordTok{as.integer}\NormalTok{(x)}
\NormalTok{x3 <-}\StringTok{ }\KeywordTok{as.numeric}\NormalTok{(x2)}
\KeywordTok{as.character}\NormalTok{(x3)}
\end{Highlighting}
\end{Shaded}

\begin{verbatim}
#> [1] "1" "0"
\end{verbatim}

What would you expect this to return?

\begin{Shaded}
\begin{Highlighting}[]
\NormalTok{x <-}\StringTok{ }\KeywordTok{c}\NormalTok{(}\OtherTok{TRUE}\NormalTok{, }\OtherTok{FALSE}\NormalTok{)}

\KeywordTok{as.integer}\NormalTok{(}\KeywordTok{as.character}\NormalTok{(x))}
\end{Highlighting}
\end{Shaded}

You can test for an ``atomic'' types of data with: \texttt{is.logical},
\texttt{is.integer}, \texttt{is.double}, \texttt{is.numeric}\footnote{\texttt{is.numeric()}
  is a general test for the ``numberliness'' of a vector and returns
  TRUE for both integer and double vectors. It is not a specific test
  for double vectors, which are often called numeric.}, and
\texttt{is.character}.

\begin{Shaded}
\begin{Highlighting}[]
\NormalTok{x <-}\StringTok{ }\KeywordTok{c}\NormalTok{(}\OtherTok{TRUE}\NormalTok{, }\OtherTok{FALSE}\NormalTok{)}

\KeywordTok{is.logical}\NormalTok{(x)}
\end{Highlighting}
\end{Shaded}

\begin{verbatim}
#> [1] TRUE
\end{verbatim}

\begin{Shaded}
\begin{Highlighting}[]
\KeywordTok{is.integer}\NormalTok{(x)}
\end{Highlighting}
\end{Shaded}

\begin{verbatim}
#> [1] FALSE
\end{verbatim}

What would you expect these to return?

\begin{Shaded}
\begin{Highlighting}[]
\NormalTok{x <-}\StringTok{ }\DecValTok{2}

\KeywordTok{is.integer}\NormalTok{(x)}
\KeywordTok{is.numeric}\NormalTok{(x)}
\KeywordTok{is.double}\NormalTok{(x)}
\end{Highlighting}
\end{Shaded}

Missing values are specified with \texttt{NA}, which is a logical vector
of length 1. \texttt{NA} will always be coerced to the correct type if
used inside \texttt{c()}, or you can create \texttt{NA}s of a specific
type with \texttt{NA\_real\_} (a double vector), \texttt{NA\_integer\_}
and \texttt{NA\_character\_}.

\subsection{Lists}\label{lists}

Lists are different from atomic vectors because their elements can be of
any type, including other lists. Lists can contain complex objects so
it's not possible to pick one visual style that works for every list.
You construct lists by using \texttt{list()} instead of \texttt{c()}:

\begin{Shaded}
\begin{Highlighting}[]
\NormalTok{x <-}\StringTok{ }\KeywordTok{list}\NormalTok{(}\DecValTok{1}\OperatorTok{:}\DecValTok{3}\NormalTok{, }\StringTok{"a"}\NormalTok{, }\KeywordTok{c}\NormalTok{(}\OtherTok{TRUE}\NormalTok{, }\OtherTok{FALSE}\NormalTok{, }\OtherTok{TRUE}\NormalTok{), }\KeywordTok{c}\NormalTok{(}\FloatTok{2.3}\NormalTok{, }\FloatTok{5.9}\NormalTok{))}
\KeywordTok{str}\NormalTok{(x)}
\end{Highlighting}
\end{Shaded}

\begin{verbatim}
#> List of 4
#>  $ : int [1:3] 1 2 3
#>  $ : chr "a"
#>  $ : logi [1:3] TRUE FALSE TRUE
#>  $ : num [1:2] 2.3 5.9
\end{verbatim}

Lists are sometimes called \textbf{recursive} vectors, because a list
can contain other lists. This makes them fundamentally different from
atomic vectors.

\begin{Shaded}
\begin{Highlighting}[]
\NormalTok{x <-}\StringTok{ }\KeywordTok{list}\NormalTok{(}\KeywordTok{list}\NormalTok{(}\KeywordTok{list}\NormalTok{(}\KeywordTok{list}\NormalTok{(}\DecValTok{1}\NormalTok{))))}
\KeywordTok{str}\NormalTok{(x)}
\end{Highlighting}
\end{Shaded}

\begin{verbatim}
#> List of 1
#>  $ :List of 1
#>   ..$ :List of 1
#>   .. ..$ :List of 1
#>   .. .. ..$ : num 1
\end{verbatim}

\begin{Shaded}
\begin{Highlighting}[]
\KeywordTok{is.recursive}\NormalTok{(x)}
\end{Highlighting}
\end{Shaded}

\begin{verbatim}
#> [1] TRUE
\end{verbatim}

\texttt{c()} will combine several lists into one. If given a combination
of atomic vectors and lists, \texttt{c()} will coerce the vectors to
lists before combining them. Compare the results of \texttt{list()} and
\texttt{c()}:

\begin{Shaded}
\begin{Highlighting}[]
\NormalTok{x <-}\StringTok{ }\KeywordTok{list}\NormalTok{(}\KeywordTok{list}\NormalTok{(}\DecValTok{1}\NormalTok{, }\DecValTok{2}\NormalTok{), }\KeywordTok{c}\NormalTok{(}\DecValTok{3}\NormalTok{, }\DecValTok{4}\NormalTok{))}
\NormalTok{y <-}\StringTok{ }\KeywordTok{c}\NormalTok{(}\KeywordTok{list}\NormalTok{(}\DecValTok{1}\NormalTok{, }\DecValTok{2}\NormalTok{), }\KeywordTok{c}\NormalTok{(}\DecValTok{3}\NormalTok{, }\DecValTok{4}\NormalTok{))}
\KeywordTok{str}\NormalTok{(x)}
\end{Highlighting}
\end{Shaded}

\begin{verbatim}
#> List of 2
#>  $ :List of 2
#>   ..$ : num 1
#>   ..$ : num 2
#>  $ : num [1:2] 3 4
\end{verbatim}

\begin{Shaded}
\begin{Highlighting}[]
\KeywordTok{str}\NormalTok{(y)}
\end{Highlighting}
\end{Shaded}

\begin{verbatim}
#> List of 4
#>  $ : num 1
#>  $ : num 2
#>  $ : num 3
#>  $ : num 4
\end{verbatim}

The \texttt{typeof()} a list is \texttt{list}. You can test for a list
with \texttt{is.list()} and coerce to a list with \texttt{as.list()}.
You can turn a list into an atomic vector with \texttt{unlist()}. If the
elements of a list have different types, \texttt{unlist()} uses the same
coercion rules as \texttt{c()}.

Lists are used to build up many of the more complicated data structures
in R. For example, both data frames (described in data frames) and
linear models objects (as produced by \texttt{lm()}) are lists

\subsection{\texorpdfstring{\texttt{NULL}}{NULL}}\label{null}

Closely related to vectors is \texttt{NULL}, a singleton object often
used to represent a vector of length 0. \texttt{NULL} is different than
\texttt{NA}. For a good explanation of the differences see
\href{https://www.r-bloggers.com/r-na-vs-null/}{this} blog post.

\subsection{Attributes}\label{attributes}

All objects can have arbitrary additional attributes, used to store
metadata about the object. Attributes can be thought of as a named
list\footnote{The reality is a little more complicated: attributes are
  actually stored in something called pairlists, which can you learn
  more about in \href{http://adv-r.had.co.nz}{Advanced R}} (with unique
names). Attributes can be accessed individually with \texttt{attr()} or
all at once (as a list) with \texttt{attributes()}.

\begin{Shaded}
\begin{Highlighting}[]
\NormalTok{a <-}\StringTok{ }\DecValTok{1}\OperatorTok{:}\DecValTok{3}
\KeywordTok{attr}\NormalTok{(a, }\StringTok{"x"}\NormalTok{) <-}\StringTok{ "abcdef"}
\KeywordTok{attr}\NormalTok{(a, }\StringTok{"y"}\NormalTok{) <-}\StringTok{ }\DecValTok{4}\OperatorTok{:}\DecValTok{6}
\KeywordTok{attr}\NormalTok{(a, }\StringTok{"z"}\NormalTok{) <-}\StringTok{ }\KeywordTok{list}\NormalTok{(}\KeywordTok{list}\NormalTok{())}
\KeywordTok{str}\NormalTok{(}\KeywordTok{attributes}\NormalTok{(a))}
\end{Highlighting}
\end{Shaded}

\begin{verbatim}
#> List of 3
#>  $ x: chr "abcdef"
#>  $ y: int [1:3] 4 5 6
#>  $ z:List of 1
#>   ..$ : list()
\end{verbatim}

The \texttt{structure()} function returns a new object with modified
attributes. Care must be taken with attributes since, by default, most
attributes are lost when modifying a vector.

\begin{Shaded}
\begin{Highlighting}[]
\KeywordTok{attributes}\NormalTok{(a[}\DecValTok{1}\NormalTok{])}
\end{Highlighting}
\end{Shaded}

\begin{verbatim}
#> NULL
\end{verbatim}

\begin{Shaded}
\begin{Highlighting}[]
\KeywordTok{attributes}\NormalTok{(}\KeywordTok{sum}\NormalTok{(a))}
\end{Highlighting}
\end{Shaded}

\begin{verbatim}
#> NULL
\end{verbatim}

The only attributes not lost are the three most important:

\begin{itemize}
\item
  Names, a character vector giving each element a name.
\item
  Dimensions, used to turn vectors into matrices and arrays.
\item
  Class, used to implement the S3 object system.
\end{itemize}

Each of these attributes has a specific accessor function to get and set
values. When working with these attributes, use \texttt{names(x)},
\texttt{dim(x)}, and \texttt{class(x)}, not \texttt{attr(x,\ "names")},
\texttt{attr(x,\ "dim")}, and \texttt{attr(x,\ "class")}.

\subsubsection{Names}\label{names}

You can name a vector in a couple\footnote{There are a couple less
  common ways. See \href{http://adv-r.had.co.nz}{Advanced R}} ways:

\begin{itemize}
\item
  When creating it:
  \texttt{x\ \textless{}-\ c(a\ =\ 1,\ b\ =\ 2,\ c\ =\ 3)}.
\item
  By modifying an existing vector in place:
  \texttt{x\ \textless{}-\ 1:3;\ names(x)\ \textless{}-\ c("a",\ "b",\ "c")}.
\end{itemize}

Named vectors a a great way to make an easy, human readable look up
table. We will see this use case extensively when we get to data
visualizations.

\subsubsection{Factors}\label{factors}

One important use of attributes is to define factors. A factor is a
vector that can contain only predefined values, and is used to store
categorical data. Factors are built on top of \textbf{integer vectors}
using two attributes: the \texttt{class}, ``factor'', which makes them
behave differently from regular integer vectors, and the
\texttt{levels}, which defines the set of allowed values.

Factors are useful when you know the possible values a variable may
take, even if you don't see all values in a given dataset. Using a
factor instead of a character vector makes it obvious when some groups
contain no observations:

\begin{Shaded}
\begin{Highlighting}[]
\NormalTok{sex_char <-}\StringTok{ }\KeywordTok{c}\NormalTok{(}\StringTok{"m"}\NormalTok{, }\StringTok{"m"}\NormalTok{, }\StringTok{"m"}\NormalTok{)}
\NormalTok{sex_factor <-}\StringTok{ }\KeywordTok{factor}\NormalTok{(sex_char, }\DataTypeTok{levels =} \KeywordTok{c}\NormalTok{(}\StringTok{"m"}\NormalTok{, }\StringTok{"f"}\NormalTok{))}

\KeywordTok{table}\NormalTok{(sex_char)}
\end{Highlighting}
\end{Shaded}

\begin{verbatim}
#> sex_char
#> m 
#> 3
\end{verbatim}

\begin{Shaded}
\begin{Highlighting}[]
\KeywordTok{table}\NormalTok{(sex_factor)}
\end{Highlighting}
\end{Shaded}

\begin{verbatim}
#> sex_factor
#> m f 
#> 3 0
\end{verbatim}

While factors look like (and often behave like) character vectors, they
are actually \textbf{integers}. Be careful when treating them like
strings. Some string methods (like \texttt{gsub()} and \texttt{grepl()})
will coerce factors to strings, while others (like \texttt{nchar()})
will throw an error, and still others (like \texttt{c()}) will use the
underlying integer values. For this reason, it is best to explicitly
convert factors to character vectors if you need string-like behaviour.

Unfortunately, many base R functions (like \texttt{read.csv()} and
\texttt{data.frame()}) automatically convert character vectors to
factors. This is suboptimal, because there's no way for those functions
to know the set of all possible levels or their optimal order. Instead,
use the argument \texttt{stringsAsFactors\ =\ FALSE} to suppress this
behaviour, and then manually convert character vectors to factors using
your knowledge of the data only when you need the behavior of factors.

Factors tend to be most useful in data visualization and table creations
where you want to report all categories but some categories may not be
present in your data, or when you want to order the categories in
somethin other than the default ordering. We will revisit factors and
there usefulness later when we study the tidyverse and in particular the
forcats package.

\subsection{Matrices and arrays}\label{matrices-and-arrays}

Adding a \texttt{dim} attribute to an atomic vector allows it to behave
like a multi-dimensional \textbf{array}. A special case of the array is
the \textbf{matrix}, which has two dimensions. Matrices are used
commonly as part of the mathematical machinery of statistics. Arrays are
much rarer, but worth being aware of.

Matrices and arrays are created with \texttt{matrix()} and
\texttt{array()}, or by using the assignment form of \texttt{dim()}:

\begin{Shaded}
\begin{Highlighting}[]
\CommentTok{# Two scalar arguments to specify rows and columns}
\NormalTok{a <-}\StringTok{ }\KeywordTok{matrix}\NormalTok{(}\DecValTok{1}\OperatorTok{:}\DecValTok{12}\NormalTok{, }\DataTypeTok{ncol =} \DecValTok{3}\NormalTok{, }\DataTypeTok{nrow =} \DecValTok{4}\NormalTok{)}
\NormalTok{a}
\end{Highlighting}
\end{Shaded}

\begin{verbatim}
#>      [,1] [,2] [,3]
#> [1,]    1    5    9
#> [2,]    2    6   10
#> [3,]    3    7   11
#> [4,]    4    8   12
\end{verbatim}

\begin{Shaded}
\begin{Highlighting}[]
\CommentTok{# One vector argument to describe all dimensions}
\NormalTok{b <-}\StringTok{ }\KeywordTok{array}\NormalTok{(}\DecValTok{1}\OperatorTok{:}\DecValTok{12}\NormalTok{, }\KeywordTok{c}\NormalTok{(}\DecValTok{2}\NormalTok{, }\DecValTok{3}\NormalTok{, }\DecValTok{2}\NormalTok{))}
\NormalTok{b}
\end{Highlighting}
\end{Shaded}

\begin{verbatim}
#> , , 1
#> 
#>      [,1] [,2] [,3]
#> [1,]    1    3    5
#> [2,]    2    4    6
#> 
#> , , 2
#> 
#>      [,1] [,2] [,3]
#> [1,]    7    9   11
#> [2,]    8   10   12
\end{verbatim}

\begin{Shaded}
\begin{Highlighting}[]
\CommentTok{# You can also modify an object in place by setting dim()}
\NormalTok{vec <-}\StringTok{ }\DecValTok{1}\OperatorTok{:}\DecValTok{12}
\NormalTok{vec}
\end{Highlighting}
\end{Shaded}

\begin{verbatim}
#>  [1]  1  2  3  4  5  6  7  8  9 10 11 12
\end{verbatim}

\begin{Shaded}
\begin{Highlighting}[]
\KeywordTok{class}\NormalTok{(vec)}
\end{Highlighting}
\end{Shaded}

\begin{verbatim}
#> [1] "integer"
\end{verbatim}

\begin{Shaded}
\begin{Highlighting}[]
\KeywordTok{dim}\NormalTok{(vec) <-}\StringTok{ }\KeywordTok{c}\NormalTok{(}\DecValTok{3}\NormalTok{, }\DecValTok{4}\NormalTok{)}
\NormalTok{vec}
\end{Highlighting}
\end{Shaded}

\begin{verbatim}
#>      [,1] [,2] [,3] [,4]
#> [1,]    1    4    7   10
#> [2,]    2    5    8   11
#> [3,]    3    6    9   12
\end{verbatim}

\begin{Shaded}
\begin{Highlighting}[]
\KeywordTok{class}\NormalTok{(vec)}
\end{Highlighting}
\end{Shaded}

\begin{verbatim}
#> [1] "matrix"
\end{verbatim}

\begin{Shaded}
\begin{Highlighting}[]
\KeywordTok{dim}\NormalTok{(vec) <-}\StringTok{ }\KeywordTok{c}\NormalTok{(}\DecValTok{3}\NormalTok{, }\DecValTok{2}\NormalTok{, }\DecValTok{2}\NormalTok{)}
\NormalTok{vec}
\end{Highlighting}
\end{Shaded}

\begin{verbatim}
#> , , 1
#> 
#>      [,1] [,2]
#> [1,]    1    4
#> [2,]    2    5
#> [3,]    3    6
#> 
#> , , 2
#> 
#>      [,1] [,2]
#> [1,]    7   10
#> [2,]    8   11
#> [3,]    9   12
\end{verbatim}

\begin{Shaded}
\begin{Highlighting}[]
\KeywordTok{class}\NormalTok{(vec)}
\end{Highlighting}
\end{Shaded}

\begin{verbatim}
#> [1] "array"
\end{verbatim}

\texttt{length()} and \texttt{names()} have high-dimensional
generalisations:

\begin{itemize}
\item
  \texttt{length()} generalises to \texttt{nrow()} and \texttt{ncol()}
  for matrices, and \texttt{dim()} for arrays.
\item
  \texttt{names()} generalises to \texttt{rownames()} and
  \texttt{colnames()} for matrices, and \texttt{dimnames()}, a list of
  character vectors, for arrays.
\end{itemize}

\texttt{c()} generalises to \texttt{cbind()} and \texttt{rbind()} for
matrices, and to \texttt{abind::abind()} for arrays. You can transpose a
matrix with \texttt{t()}; the generalised equivalent for arrays is
\texttt{aperm()}.

You can test if an object is a matrix or array using
\texttt{is.matrix()} and \texttt{is.array()}, or by looking at the
length of the \texttt{dim()}. \texttt{as.matrix()} and
\texttt{as.array()} make it easy to turn an existing vector into a
matrix or array.

Vectors are not the only 1-dimensional data structure. You can have
matrices with a single row or single column, or arrays with a single
dimension. They may print similarly, but will behave differently. The
differences aren't too important, but it's useful to know they exist in
case you get strange output from a function (\texttt{tapply()} is a
frequent offender). As always, use \texttt{str()} to reveal the
differences.

Matrices and arrays are most useful for mathematical calculations
(particularly when fitting models); lists are a better fit for most
other programming tasks in R.

\subsection{Data Frames}\label{data-frames}

A data frame is the most common way of storing data in R, and if used
systematically makes data analysis easier. Under the hood, a data frame
is a list of equal-length vectors. This makes it a 2-dimensional
structure, so it shares properties of both the matrix and the list. This
means that a data frame has \texttt{names()}, \texttt{colnames()}, and
\texttt{rownames()}, although \texttt{names()} and \texttt{colnames()}
are the same thing. The \texttt{length()} of a data frame is the length
of the underlying list and so is the same as \texttt{ncol()};
\texttt{nrow()} gives the number of rows. You can subset a data frame
like a 1d structure (where it behaves like a list), or a 2d structure
(where it behaves like a matrix), we will discuss this further when we
discuss subsetting.

\subsubsection{Creation}\label{creation}

You create a data frame using \texttt{data.frame()}, which takes named
vectors as input:

\begin{Shaded}
\begin{Highlighting}[]
\NormalTok{df <-}\StringTok{ }\KeywordTok{data.frame}\NormalTok{(}\DataTypeTok{x =} \DecValTok{1}\OperatorTok{:}\DecValTok{3}\NormalTok{, }\DataTypeTok{y =} \KeywordTok{c}\NormalTok{(}\StringTok{"a"}\NormalTok{, }\StringTok{"b"}\NormalTok{, }\StringTok{"c"}\NormalTok{))}
\KeywordTok{str}\NormalTok{(df)}
\end{Highlighting}
\end{Shaded}

\begin{verbatim}
#> 'data.frame':    3 obs. of  2 variables:
#>  $ x: int  1 2 3
#>  $ y: Factor w/ 3 levels "a","b","c": 1 2 3
\end{verbatim}

Beware \texttt{data.frame()}'s default behaviour which turns strings
into factors. Use \texttt{stringsAsFactors\ =\ FALSE} to suppress this
behaviour:

\begin{Shaded}
\begin{Highlighting}[]
\NormalTok{df <-}\StringTok{ }\KeywordTok{data.frame}\NormalTok{(}
  \DataTypeTok{x =} \DecValTok{1}\OperatorTok{:}\DecValTok{3}\NormalTok{,}
  \DataTypeTok{y =} \KeywordTok{c}\NormalTok{(}\StringTok{"a"}\NormalTok{, }\StringTok{"b"}\NormalTok{, }\StringTok{"c"}\NormalTok{),}
  \DataTypeTok{stringsAsFactors =} \OtherTok{FALSE}\NormalTok{)}
\KeywordTok{str}\NormalTok{(df)}
\end{Highlighting}
\end{Shaded}

\begin{verbatim}
#> 'data.frame':    3 obs. of  2 variables:
#>  $ x: int  1 2 3
#>  $ y: chr  "a" "b" "c"
\end{verbatim}

\subsubsection{Testing and coercion}\label{testing-and-coercion}

Because a \texttt{data.frame} is an S3 class, its type reflects the
underlying vector used to build it: the list. To check if an object is a
data frame, use \texttt{is.data.frame()}:

\begin{Shaded}
\begin{Highlighting}[]
\KeywordTok{is.data.frame}\NormalTok{(df)}
\end{Highlighting}
\end{Shaded}

\begin{verbatim}
#> [1] TRUE
\end{verbatim}

You can coerce an object to a data frame with \texttt{as.data.frame()}:

\begin{itemize}
\item
  A vector will create a one-column data frame.
\item
  A list will create one column for each element; it's an error if
  they're not all the same length.
\item
  A matrix will create a data frame with the same number of columns and
  rows as the matrix.
\end{itemize}

The automatic coercion that causes the most problems is if you select a
single column of a data.frame. R will coerce the column to an atomic
vector, which generally is not what you want\footnote{We'll revisit this
  when we get into R for Data Science and discuss tibbles}.

\begin{Shaded}
\begin{Highlighting}[]
\NormalTok{(x1 <-}\StringTok{ }\NormalTok{df[, }\StringTok{"x"}\NormalTok{])}
\end{Highlighting}
\end{Shaded}

\begin{verbatim}
#> [1] 1 2 3
\end{verbatim}

\begin{Shaded}
\begin{Highlighting}[]
\KeywordTok{str}\NormalTok{(x1)}
\end{Highlighting}
\end{Shaded}

\begin{verbatim}
#>  int [1:3] 1 2 3
\end{verbatim}

\begin{Shaded}
\begin{Highlighting}[]
\NormalTok{(x2 <-}\StringTok{ }\NormalTok{df[, }\StringTok{"y"}\NormalTok{, }\DataTypeTok{drop =} \OtherTok{FALSE}\NormalTok{])}
\end{Highlighting}
\end{Shaded}

\begin{verbatim}
#>   y
#> 1 a
#> 2 b
#> 3 c
\end{verbatim}

\begin{Shaded}
\begin{Highlighting}[]
\KeywordTok{str}\NormalTok{(x2)}
\end{Highlighting}
\end{Shaded}

\begin{verbatim}
#> 'data.frame':    3 obs. of  1 variable:
#>  $ y: chr  "a" "b" "c"
\end{verbatim}

\subsubsection{Combining data frames}\label{combining-data-frames}

You can combine data frames using \texttt{cbind()} and \texttt{rbind()}:

\begin{Shaded}
\begin{Highlighting}[]
\KeywordTok{cbind}\NormalTok{(df, }\KeywordTok{data.frame}\NormalTok{(}\DataTypeTok{z =} \DecValTok{3}\OperatorTok{:}\DecValTok{1}\NormalTok{))}
\end{Highlighting}
\end{Shaded}

\begin{verbatim}
#>   x y z
#> 1 1 a 3
#> 2 2 b 2
#> 3 3 c 1
\end{verbatim}

\begin{Shaded}
\begin{Highlighting}[]
\KeywordTok{rbind}\NormalTok{(df, }\KeywordTok{data.frame}\NormalTok{(}\DataTypeTok{x =} \DecValTok{10}\NormalTok{, }\DataTypeTok{y =} \StringTok{"z"}\NormalTok{))}
\end{Highlighting}
\end{Shaded}

\begin{verbatim}
#>    x y
#> 1  1 a
#> 2  2 b
#> 3  3 c
#> 4 10 z
\end{verbatim}

When combining column-wise, the number of rows must match, but row names
are ignored. When combining row-wise, both the number and names of
columns must match. Use \texttt{dplyr::bind\_rows()},
\texttt{data.table::rbindlist()}, or similar to combine data frames that
don't have the same columns.

It's a common mistake to try and create a data frame by
\texttt{cbind()}ing vectors together. This is unlikely to do what you
want because \texttt{cbind()} will create a matrix unless one of the
arguments is already a data frame. Instead use \texttt{data.frame()}
directly:

\begin{Shaded}
\begin{Highlighting}[]
\CommentTok{# This is always a mistake}
\NormalTok{bad <-}\StringTok{ }\KeywordTok{data.frame}\NormalTok{(}\KeywordTok{cbind}\NormalTok{(}\DataTypeTok{a =} \DecValTok{1}\OperatorTok{:}\DecValTok{2}\NormalTok{, }\DataTypeTok{b =} \KeywordTok{c}\NormalTok{(}\StringTok{"a"}\NormalTok{, }\StringTok{"b"}\NormalTok{)))}
\KeywordTok{str}\NormalTok{(bad)}
\end{Highlighting}
\end{Shaded}

\begin{verbatim}
#> 'data.frame':    2 obs. of  2 variables:
#>  $ a: Factor w/ 2 levels "1","2": 1 2
#>  $ b: Factor w/ 2 levels "a","b": 1 2
\end{verbatim}

\begin{Shaded}
\begin{Highlighting}[]
\NormalTok{good <-}\StringTok{ }\KeywordTok{data.frame}\NormalTok{(}\DataTypeTok{a =} \DecValTok{1}\OperatorTok{:}\DecValTok{2}\NormalTok{, }\DataTypeTok{b =} \KeywordTok{c}\NormalTok{(}\StringTok{"a"}\NormalTok{, }\StringTok{"b"}\NormalTok{))}
\KeywordTok{str}\NormalTok{(good)}
\end{Highlighting}
\end{Shaded}

\begin{verbatim}
#> 'data.frame':    2 obs. of  2 variables:
#>  $ a: int  1 2
#>  $ b: Factor w/ 2 levels "a","b": 1 2
\end{verbatim}

\subsubsection{List and matrix columns}\label{list-and-matrix-columns}

Since a data frame is a list of vectors, it is possible for a data frame
to have a column that is a list. This is a powerful technique because a
list can contain any other R object. This means that you can have a
column of data frames, or model objects, or even functions! We will see
this again when we discuss tidy data.

\begin{Shaded}
\begin{Highlighting}[]
\NormalTok{df <-}\StringTok{ }\KeywordTok{data.frame}\NormalTok{(}\DataTypeTok{x =} \DecValTok{1}\OperatorTok{:}\DecValTok{3}\NormalTok{)}
\NormalTok{df}\OperatorTok{$}\NormalTok{y <-}\StringTok{ }\KeywordTok{list}\NormalTok{(}\DecValTok{1}\OperatorTok{:}\DecValTok{2}\NormalTok{, }\DecValTok{1}\OperatorTok{:}\DecValTok{3}\NormalTok{, }\DecValTok{1}\OperatorTok{:}\DecValTok{4}\NormalTok{)}
\NormalTok{df}
\end{Highlighting}
\end{Shaded}

\begin{verbatim}
#>   x          y
#> 1 1       1, 2
#> 2 2    1, 2, 3
#> 3 3 1, 2, 3, 4
\end{verbatim}

However, when a list is given to \texttt{data.frame()}, it tries to put
each item of the list into its own column, so this fails:

\begin{Shaded}
\begin{Highlighting}[]
\KeywordTok{data.frame}\NormalTok{(}\DataTypeTok{x =} \DecValTok{1}\OperatorTok{:}\DecValTok{3}\NormalTok{, }\DataTypeTok{y =} \KeywordTok{list}\NormalTok{(}\DecValTok{1}\OperatorTok{:}\DecValTok{2}\NormalTok{, }\DecValTok{1}\OperatorTok{:}\DecValTok{3}\NormalTok{, }\DecValTok{1}\OperatorTok{:}\DecValTok{4}\NormalTok{))}
\end{Highlighting}
\end{Shaded}

\begin{verbatim}
#> Error in (function (..., row.names = NULL, check.rows = FALSE, check.names = TRUE, : arguments imply differing number of rows: 2, 3, 4
\end{verbatim}

A workaround is to use \texttt{I()}, which causes \texttt{data.frame()}
to treat the list as one unit:

\begin{Shaded}
\begin{Highlighting}[]
\NormalTok{dfl <-}\StringTok{ }\KeywordTok{data.frame}\NormalTok{(}\DataTypeTok{x =} \DecValTok{1}\OperatorTok{:}\DecValTok{3}\NormalTok{, }\DataTypeTok{y =} \KeywordTok{I}\NormalTok{(}\KeywordTok{list}\NormalTok{(}\DecValTok{1}\OperatorTok{:}\DecValTok{2}\NormalTok{, }\DecValTok{1}\OperatorTok{:}\DecValTok{3}\NormalTok{, }\DecValTok{1}\OperatorTok{:}\DecValTok{4}\NormalTok{)))}
\KeywordTok{str}\NormalTok{(dfl)}
\end{Highlighting}
\end{Shaded}

\begin{verbatim}
#> 'data.frame':    3 obs. of  2 variables:
#>  $ x: int  1 2 3
#>  $ y:List of 3
#>   ..$ : int  1 2
#>   ..$ : int  1 2 3
#>   ..$ : int  1 2 3 4
#>   ..- attr(*, "class")= chr "AsIs"
\end{verbatim}

\texttt{I()} adds the \texttt{AsIs} class to its input, but this can
usually be safely ignored.

Similarly, it's also possible to have a column of a data frame that's a
matrix or array, as long as the number of rows matches the data frame:

\begin{Shaded}
\begin{Highlighting}[]
\NormalTok{dfm <-}\StringTok{ }\KeywordTok{data.frame}\NormalTok{(}\DataTypeTok{x =} \DecValTok{1}\OperatorTok{:}\DecValTok{3} \OperatorTok{*}\StringTok{ }\DecValTok{10}\NormalTok{, }\DataTypeTok{y =} \KeywordTok{I}\NormalTok{(}\KeywordTok{matrix}\NormalTok{(}\DecValTok{1}\OperatorTok{:}\DecValTok{9}\NormalTok{, }\DataTypeTok{nrow =} \DecValTok{3}\NormalTok{)))}
\KeywordTok{str}\NormalTok{(dfm)}
\end{Highlighting}
\end{Shaded}

\begin{verbatim}
#> 'data.frame':    3 obs. of  2 variables:
#>  $ x: num  10 20 30
#>  $ y: 'AsIs' int [1:3, 1:3] 1 2 3 4 5 6 7 8 9
\end{verbatim}

Use list and array columns with caution. Many functions that work with
data frames assume that all columns are atomic vectors, and the printed
display can be confusing.

\begin{Shaded}
\begin{Highlighting}[]
\NormalTok{dfl[}\DecValTok{2}\NormalTok{, ]}
\end{Highlighting}
\end{Shaded}

\begin{verbatim}
#>   x       y
#> 2 2 1, 2, 3
\end{verbatim}

\begin{Shaded}
\begin{Highlighting}[]
\NormalTok{dfm[}\DecValTok{2}\NormalTok{, ]}
\end{Highlighting}
\end{Shaded}

\begin{verbatim}
#>    x y.1 y.2 y.3
#> 2 20   2   5   8
\end{verbatim}

\section{Functions}\label{functions}

\subsection{Functional Programming}\label{functional-programming}

\subsection{Functionals}\label{functionals}

\subsection{Function operators}\label{function-operators}

\section{Environments}\label{environments}

Scoping

\bibliography{book.bib,packages.bib}


\end{document}
