\documentclass[]{book}
\usepackage{lmodern}
\usepackage{amssymb,amsmath}
\usepackage{ifxetex,ifluatex}
\usepackage{fixltx2e} % provides \textsubscript
\ifnum 0\ifxetex 1\fi\ifluatex 1\fi=0 % if pdftex
  \usepackage[T1]{fontenc}
  \usepackage[utf8]{inputenc}
\else % if luatex or xelatex
  \ifxetex
    \usepackage{mathspec}
  \else
    \usepackage{fontspec}
  \fi
  \defaultfontfeatures{Ligatures=TeX,Scale=MatchLowercase}
\fi
% use upquote if available, for straight quotes in verbatim environments
\IfFileExists{upquote.sty}{\usepackage{upquote}}{}
% use microtype if available
\IfFileExists{microtype.sty}{%
\usepackage{microtype}
\UseMicrotypeSet[protrusion]{basicmath} % disable protrusion for tt fonts
}{}
\usepackage[margin=1in]{geometry}
\usepackage{hyperref}
\hypersetup{unicode=true,
            pdftitle={Modern R in a Corporate Environment},
            pdfauthor={Brian Davis},
            pdfborder={0 0 0},
            breaklinks=true}
\urlstyle{same}  % don't use monospace font for urls
\usepackage{natbib}
\bibliographystyle{apalike}
\usepackage{color}
\usepackage{fancyvrb}
\newcommand{\VerbBar}{|}
\newcommand{\VERB}{\Verb[commandchars=\\\{\}]}
\DefineVerbatimEnvironment{Highlighting}{Verbatim}{commandchars=\\\{\}}
% Add ',fontsize=\small' for more characters per line
\usepackage{framed}
\definecolor{shadecolor}{RGB}{248,248,248}
\newenvironment{Shaded}{\begin{snugshade}}{\end{snugshade}}
\newcommand{\KeywordTok}[1]{\textcolor[rgb]{0.13,0.29,0.53}{\textbf{#1}}}
\newcommand{\DataTypeTok}[1]{\textcolor[rgb]{0.13,0.29,0.53}{#1}}
\newcommand{\DecValTok}[1]{\textcolor[rgb]{0.00,0.00,0.81}{#1}}
\newcommand{\BaseNTok}[1]{\textcolor[rgb]{0.00,0.00,0.81}{#1}}
\newcommand{\FloatTok}[1]{\textcolor[rgb]{0.00,0.00,0.81}{#1}}
\newcommand{\ConstantTok}[1]{\textcolor[rgb]{0.00,0.00,0.00}{#1}}
\newcommand{\CharTok}[1]{\textcolor[rgb]{0.31,0.60,0.02}{#1}}
\newcommand{\SpecialCharTok}[1]{\textcolor[rgb]{0.00,0.00,0.00}{#1}}
\newcommand{\StringTok}[1]{\textcolor[rgb]{0.31,0.60,0.02}{#1}}
\newcommand{\VerbatimStringTok}[1]{\textcolor[rgb]{0.31,0.60,0.02}{#1}}
\newcommand{\SpecialStringTok}[1]{\textcolor[rgb]{0.31,0.60,0.02}{#1}}
\newcommand{\ImportTok}[1]{#1}
\newcommand{\CommentTok}[1]{\textcolor[rgb]{0.56,0.35,0.01}{\textit{#1}}}
\newcommand{\DocumentationTok}[1]{\textcolor[rgb]{0.56,0.35,0.01}{\textbf{\textit{#1}}}}
\newcommand{\AnnotationTok}[1]{\textcolor[rgb]{0.56,0.35,0.01}{\textbf{\textit{#1}}}}
\newcommand{\CommentVarTok}[1]{\textcolor[rgb]{0.56,0.35,0.01}{\textbf{\textit{#1}}}}
\newcommand{\OtherTok}[1]{\textcolor[rgb]{0.56,0.35,0.01}{#1}}
\newcommand{\FunctionTok}[1]{\textcolor[rgb]{0.00,0.00,0.00}{#1}}
\newcommand{\VariableTok}[1]{\textcolor[rgb]{0.00,0.00,0.00}{#1}}
\newcommand{\ControlFlowTok}[1]{\textcolor[rgb]{0.13,0.29,0.53}{\textbf{#1}}}
\newcommand{\OperatorTok}[1]{\textcolor[rgb]{0.81,0.36,0.00}{\textbf{#1}}}
\newcommand{\BuiltInTok}[1]{#1}
\newcommand{\ExtensionTok}[1]{#1}
\newcommand{\PreprocessorTok}[1]{\textcolor[rgb]{0.56,0.35,0.01}{\textit{#1}}}
\newcommand{\AttributeTok}[1]{\textcolor[rgb]{0.77,0.63,0.00}{#1}}
\newcommand{\RegionMarkerTok}[1]{#1}
\newcommand{\InformationTok}[1]{\textcolor[rgb]{0.56,0.35,0.01}{\textbf{\textit{#1}}}}
\newcommand{\WarningTok}[1]{\textcolor[rgb]{0.56,0.35,0.01}{\textbf{\textit{#1}}}}
\newcommand{\AlertTok}[1]{\textcolor[rgb]{0.94,0.16,0.16}{#1}}
\newcommand{\ErrorTok}[1]{\textcolor[rgb]{0.64,0.00,0.00}{\textbf{#1}}}
\newcommand{\NormalTok}[1]{#1}
\usepackage{longtable,booktabs}
\usepackage{graphicx,grffile}
\makeatletter
\def\maxwidth{\ifdim\Gin@nat@width>\linewidth\linewidth\else\Gin@nat@width\fi}
\def\maxheight{\ifdim\Gin@nat@height>\textheight\textheight\else\Gin@nat@height\fi}
\makeatother
% Scale images if necessary, so that they will not overflow the page
% margins by default, and it is still possible to overwrite the defaults
% using explicit options in \includegraphics[width, height, ...]{}
\setkeys{Gin}{width=\maxwidth,height=\maxheight,keepaspectratio}
\IfFileExists{parskip.sty}{%
\usepackage{parskip}
}{% else
\setlength{\parindent}{0pt}
\setlength{\parskip}{6pt plus 2pt minus 1pt}
}
\setlength{\emergencystretch}{3em}  % prevent overfull lines
\providecommand{\tightlist}{%
  \setlength{\itemsep}{0pt}\setlength{\parskip}{0pt}}
\setcounter{secnumdepth}{5}
% Redefines (sub)paragraphs to behave more like sections
\ifx\paragraph\undefined\else
\let\oldparagraph\paragraph
\renewcommand{\paragraph}[1]{\oldparagraph{#1}\mbox{}}
\fi
\ifx\subparagraph\undefined\else
\let\oldsubparagraph\subparagraph
\renewcommand{\subparagraph}[1]{\oldsubparagraph{#1}\mbox{}}
\fi

%%% Use protect on footnotes to avoid problems with footnotes in titles
\let\rmarkdownfootnote\footnote%
\def\footnote{\protect\rmarkdownfootnote}

%%% Change title format to be more compact
\usepackage{titling}

% Create subtitle command for use in maketitle
\newcommand{\subtitle}[1]{
  \posttitle{
    \begin{center}\large#1\end{center}
    }
}

\setlength{\droptitle}{-2em}

  \title{Modern R in a Corporate Environment}
    \pretitle{\vspace{\droptitle}\centering\huge}
  \posttitle{\par}
  \subtitle{R course developed for the office}
  \author{Brian Davis}
    \preauthor{\centering\large\emph}
  \postauthor{\par}
      \predate{\centering\large\emph}
  \postdate{\par}
    \date{2018-10-22}

\usepackage{booktabs}
\usepackage{longtable}
\usepackage{framed,color}
\definecolor{shadecolor}{RGB}{248,248,248}

\ifxetex
  \usepackage{letltxmacro}
  \setlength{\XeTeXLinkMargin}{1pt}
  \LetLtxMacro\SavedIncludeGraphics\includegraphics
  \def\includegraphics#1#{% #1 catches optional stuff (star/opt. arg.)
    \IncludeGraphicsAux{#1}%
  }%
  \newcommand*{\IncludeGraphicsAux}[2]{%
    \XeTeXLinkBox{%
      \SavedIncludeGraphics#1{#2}%
    }%
  }%
\fi

\newenvironment{rmdblock}[1]
  {\begin{shaded*}
  \begin{itemize}
  \renewcommand{\labelitemi}{
    \raisebox{-.7\height}[0pt][0pt]{
      {\setkeys{Gin}{width=3em,keepaspectratio}\includegraphics{images/#1}}
    }
  }
  \item
  }
  {
  \end{itemize}
  \end{shaded*}
  }
\newenvironment{rmdnote}
  {\begin{rmdblock}{note}}
  {\end{rmdblock}}
\newenvironment{rmdtip}
  {\begin{rmdblock}{tip}}
  {\end{rmdblock}}
\newenvironment{rmdwarning}
  {\begin{rmdblock}{warning}}
  {\end{rmdblock}}
\newenvironment{rmdcaution}
  {\begin{rmdblock}{caution}}
  {\end{rmdblock}}
  \newenvironment{rmdimportant}
  {\begin{rmdblock}{important}}
  {\end{rmdblock}}
  
\usepackage{amsthm}
\makeatletter
\def\thm@space@setup{%
  \thm@preskip=8pt plus 2pt minus 4pt
  \thm@postskip=\thm@preskip
}
\makeatother

\usepackage{amsthm}
\newtheorem{theorem}{Theorem}[chapter]
\newtheorem{lemma}{Lemma}[chapter]
\theoremstyle{definition}
\newtheorem{definition}{Definition}[chapter]
\newtheorem{corollary}{Corollary}[chapter]
\newtheorem{proposition}{Proposition}[chapter]
\theoremstyle{definition}
\newtheorem{example}{Example}[chapter]
\theoremstyle{definition}
\newtheorem{exercise}{Exercise}[chapter]
\theoremstyle{remark}
\newtheorem*{remark}{Remark}
\newtheorem*{solution}{Solution}
\let\BeginKnitrBlock\begin \let\EndKnitrBlock\end
\begin{document}
\maketitle

{
\setcounter{tocdepth}{1}
\tableofcontents
}
\chapter*{Welcome}\label{welcome}
\addcontentsline{toc}{chapter}{Welcome}

\begin{quote}
Something that will make life easier in the long-run can be the most
difficult thing to do today. For coders, prioritising the long term may
involve an overhaul of current practice and the learning of a new skill.
\end{quote}

This is the course notes for our class. This course will teach you how
to do data science with R. You'll learn the basics of R and then we'll
go through \href{http://r4ds.had.co.nz/index.html}{R for Data Science}
by Garrett Grolemund \& Hadley Wickham. You'll learn how to get your
data into R, get it into the most useful structure, transform it,
visualize it and communicate out your results. We'll mix in various
topics from our current workload as well as some unique challenges of
working in a corporate environment.

Most of these are the skills that allow data science to happen, and here
you will find the best practices for doing each of these things with R.
You'll learn how to use the grammar of graphics, literate programming,
and reproducible research to save time and reduce errors.

We will build the tools to make our work easier and more streamlined
together.

\part{Preamble}\label{part-preamble}

\chapter{Introduction}\label{preamble-intro}

\section{Course Philosophy}\label{course-philosophy}

\begin{quote}
``The best programs are written so that computing machines can perform
them quickly and so that human beings can understand them clearly. A
programmer is ideally an essayist who works with traditional aesthetic
and literary forms as well as mathematical concepts, to communicate the
way that an algorithm works and to convince a reader that the results
will be correct.''

--- Donald Knuth
\end{quote}

\subsection{Reproducible Research
Approach}\label{reproducible-research-approach}

\href{https://www.coursera.org/learn/reproducible-research/lecture/FvOGB/what-is-reproducible-research-about}{What
is Reproducible Research About?}

Reproducible research is the idea that data analyses, and more
generally, scientific claims, are published with their data and software
code so that others may verify the findings and build upon them. There
are two basic reasons to be concerned about making your research
reproducible. The first is \emph{to show evidence of the correctness of
your results}. The second reason to aspire to reproducibility is
\emph{to enable others to make use of our methods and results}.

Modern challenges of reproducibility in research, particularly
computational reproducibility, have produced a lot of discussion in
papers, blogs and videos, some of which are listed
\href{http://ropensci.github.io/reproducibility-guide/sections/references/}{here}
and \href{https://reproducibleresearch.net/}{here}.

\begin{quote}
Conclusions in experimental psychology often are the result of null
hypothesis significance testing. Unfortunately, there is evidence ((from
eight major psychology journals published between 1985 and 2013) that
roughly half of all published empirical psychology articles contain at
least one inconsistent p-value, and around one in eight articles contain
a grossly inconsistent p-value that makes a non-significant result seem
significant, or vice versa.
\href{https://mbnuijten.com/statcheck/}{statscheck} and
\href{http://blog.revolutionanalytics.com/2016/10/statcheck.html}{here}
\end{quote}

\begin{quote}
``A key component of scientific communication is sufficient information
for other researchers in the field to reproduce published findings. For
computational and data-enabled research, this has often been interpreted
to mean making available the raw data from which results were generated,
the computer code that generated the findings, and any additional
information needed such as workflows and input parameters. Many journals
are revising author guidelines to include data and code availability. We
chose a random sample of 204 scientific papers published in the journal
\textbf{Science} after the implementation of their policy in February
2011. We found that were able to reproduce the findings for 26\%.''
\href{http://www.pnas.org/content/115/11/2584}{Proceedings of the
National Academy of Sciences of the United States of America}
\end{quote}

\begin{quote}
``Starting September 1 2016, JASA ACS will require code and data as a
minimum standard for reproducibility of statistical scientific
research.''
\href{https://magazine.amstat.org/blog/2016/07/01/jasa-reproducible16/}{JASA}
\end{quote}

\subsection{FDA Validation}\label{fda-validation}

\begin{quote}
``Establishing documented evidence which provides a high degree of
assurance that a specific process will consistently produce a product
meeting its predetermined specifications and quality attributes.''
-Validation as defined by the FDA in \textbf{Validation of Systems for
21 CFR Part 11 Compliance}
\end{quote}

\subsection{The SAS Myth}\label{the-sas-myth}

Contrary to what we hear the FDA does not require SAS to be used
\emph{EVER}. There are instances that you have to deliver data in XPORT
format though which is open and implemented in many programming
languages.

\begin{quote}
``FDA does not require use of any specific software for statistical
analyses, and statistical software is not explicitly discussed in Title
21 of the Code of Federal Regulations {[}e.g., in 21CFR part 11{]}.
However, the software package(s) used for statistical analyses should be
fully documented in the submission, including version and build
identification. As noted in the FDA guidance, E9 Statistical Principles
for Clinical Trials''
\href{https://www.fda.gov/downloads/forindustry/datastandards/studydatastandards/ucm587506.pdf}{FDA
Statistical Software Clarifying Statement}
\end{quote}

Good \href{http://blog.revolutionanalytics.com/2017/06/r-fda.html}{write
up} with links to several FDA talks on the
\href{https://thomaswdinsmore.com/2014/12/01/sas-versus-r-part-1/}{subject}.

\section{Prerequisites}\label{prerequisites}

\begin{itemize}
\tightlist
\item
  We will assume you have minimal experience and knowledge of R
\item
  IT should have installed:

  \begin{itemize}
  \tightlist
  \item
    \href{https://cran.r-project.org/}{R} version 3.5.1
  \item
    \href{https://www.rstudio.com/products/rstudio/download/\#download}{RStudio}
    version 1.1
  \item
    \href{https://miktex.org/}{MiTeX}
  \item
    \href{https://cran.r-project.org/bin/windows/Rtools/}{RTools}
    version 3.4
  \end{itemize}
\item
  We will install other dependencies throughout the course.
\end{itemize}

\section{Content}\label{content}

It is impossible to become an expert in R in only one course even a
multi-week one. Our aim is at gaining a wide understanding on many
aspects of R as used in a corporate / production environment. It will
roughly be based on \href{http://r4ds.had.co.nz}{R for Data Science}.
While this is an \emph{excellent} resource it does not cover much of
what we will need on a routine basis. Some external resources will be
referred to in this book for you to be able to deepen what you would
have learned in this course.

We will focus most of our attention to the \emph{tidyverse} family of
packages for data analysis. The \emph{tidyverse} is an opinionated
\href{https://www.tidyverse.org/packages/}{collection of R packages}
designed for data science. All packages share an underlying design
philosophy, grammar, and data structures.

This is your course so if you feel we need to hit an area deeper, or add
content based on a current need, let me know an we will work to adjust
it.

The \textbf{rough} topic list of the course:

\begin{enumerate}
\def\labelenumi{\arabic{enumi}.}
\tightlist
\item
  Good programming practices
\item
  Basics of R Programming
\item
  Importing / Exporting Data
\item
  Tidying Data
\item
  Visualizing Data
\item
  Functions
\item
  Strings
\item
  Dates and Time
\item
  Communicating Results
\item
  Iteration
\end{enumerate}

\section{Structure}\label{structure}

My current thoughts are to meet an hour a week and discuss a topic. We
will not be going strictly through the R4DS, but will use it as our
foundation into the topic at hand. Then give some exercises due for the
next week which we go over the solutions. We will incorporate these
exercises into an R package(s?) so we will have a collection of useful
reusable code for the future.

Open to other ideas as we go along.

I'm going to try to keep the assignments related to our current work so
we can work on the class during work hours. Bring what you are working
on and we will see how we can fit it into the class.

\chapter{Good practices}\label{good-practices}

\begin{quote}
``When you write a program, think of it primarily as a work of
literature. You're trying to write something that human beings are going
to read. Don't think of it primarily as something a computer is going to
follow. The more effective you are at making your program readable, the
more effective it's going to be: You'll understand it today, you'll
understand it next week, and your successors who are going to maintain
and modify it will understand it.''

-- Donald Knuth
\end{quote}

\section{Coding style}\label{coding-style}

Good coding style is like correct punctuation: you can manage without
it, butitsuremakesthingseasiertoread. When I answer questions; first, I
see if think I can answer the question, secondly, I check the coding
style of the question and if the code is too difficult to read, I just
move on. Please make your code readable by following e.g.
\href{http://style.tidyverse.org/}{this coding style} (most examples
below come from this guide).

\begin{quote}
``To become ssignificantly more reliable, code must become more
transparent. In particular, nested conditions and loops must be viewed
with great suspicion. Complicated control flows confuse programmers.
\textbf{Messy code often hides bugs}.''

--- Bjarne Stroustrup
\end{quote}

\subsection{Comments}\label{comments}

In code, use comments to explain the ``why'' not the ``what'' or
``how''. Each line of a comment should begin with the comment symbol and
a single space: \texttt{\#}.

\BeginKnitrBlock{rmdtip}
Use commented lines of - to break up your file into easily readable
chunks and to create a code outline in RStudio
\EndKnitrBlock{rmdtip}

\subsection{Naming}\label{naming}

\begin{quote}
There are only two hard things in Computer Science: cache invalidation
and naming things.

-- Phil Karlton
\end{quote}

Names are not limited to 8 characters as in some other languages,
however they are case sensitive. Be smart with your naming; be
descriptive yet concise. Think about how your names will show up in auto
complete.

Throughout the course we will point out some standard naming conventions
that are used in R (and other languages). (Ex. \texttt{i} and \texttt{j}
as row and column indices)

\begin{Shaded}
\begin{Highlighting}[]
\CommentTok{# Good}
\NormalTok{average_height <-}\StringTok{ }\KeywordTok{mean}\NormalTok{((feet }\OperatorTok{/}\StringTok{ }\DecValTok{12}\NormalTok{) }\OperatorTok{+}\StringTok{ }\NormalTok{inches)}
\KeywordTok{plot}\NormalTok{(mtcars}\OperatorTok{$}\NormalTok{disp, mtcars}\OperatorTok{$}\NormalTok{mpg)}

\CommentTok{# Bad}
\NormalTok{ah<-}\KeywordTok{mean}\NormalTok{(x}\OperatorTok{/}\DecValTok{12}\OperatorTok{+}\NormalTok{y)}
\KeywordTok{plot}\NormalTok{(mtcars[, }\DecValTok{3}\NormalTok{], mtcars[, }\DecValTok{1}\NormalTok{])}
\end{Highlighting}
\end{Shaded}

\subsection{Spacing}\label{spacing}

Put a space before and after \texttt{=} when naming arguments in
function calls. Most infix operators (\texttt{==}, \texttt{+},
\texttt{-}, \texttt{\textless{}-}, etc.) are also surrounded by spaces,
except those with relatively high precedence: \texttt{\^{}}, \texttt{:},
\texttt{::}, and \texttt{:::}. Always put a space after a comma, and
never before (just like in regular English).

\begin{Shaded}
\begin{Highlighting}[]
\CommentTok{# Good}
\NormalTok{average <-}\StringTok{ }\KeywordTok{mean}\NormalTok{((feet }\OperatorTok{/}\StringTok{ }\DecValTok{12}\NormalTok{) }\OperatorTok{+}\StringTok{ }\NormalTok{inches, }\DataTypeTok{na.rm =} \OtherTok{TRUE}\NormalTok{)}
\KeywordTok{sqrt}\NormalTok{(x}\OperatorTok{^}\DecValTok{2} \OperatorTok{+}\StringTok{ }\NormalTok{y}\OperatorTok{^}\DecValTok{2}\NormalTok{)}
\NormalTok{x <-}\StringTok{ }\DecValTok{1}\OperatorTok{:}\DecValTok{10}
\NormalTok{base}\OperatorTok{::}\NormalTok{sum}

\CommentTok{# Bad}
\NormalTok{average<-}\KeywordTok{mean}\NormalTok{(feet}\OperatorTok{/}\DecValTok{12}\OperatorTok{+}\NormalTok{inches,}\DataTypeTok{na.rm=}\OtherTok{TRUE}\NormalTok{)}
\KeywordTok{sqrt}\NormalTok{(x }\OperatorTok{^}\StringTok{ }\DecValTok{2} \OperatorTok{+}\StringTok{ }\NormalTok{y }\OperatorTok{^}\StringTok{ }\DecValTok{2}\NormalTok{)}
\NormalTok{x <-}\StringTok{ }\DecValTok{1} \OperatorTok{:}\StringTok{ }\DecValTok{10}
\NormalTok{base }\OperatorTok{::}\StringTok{ }\NormalTok{sum}
\end{Highlighting}
\end{Shaded}

\subsection{Indenting}\label{indenting}

Curly braces, \texttt{\{\}}, define the the most important hierarchy of
R code. To make this hierarchy easy to see, always indent the code
inside \texttt{\{\}} by two spaces.

\begin{Shaded}
\begin{Highlighting}[]
\CommentTok{# Good}
\ControlFlowTok{if}\NormalTok{ (y }\OperatorTok{<}\StringTok{ }\DecValTok{0} \OperatorTok{&&}\StringTok{ }\NormalTok{debug) \{}
  \KeywordTok{message}\NormalTok{(}\StringTok{"y is negative"}\NormalTok{)}
\NormalTok{\}}

\ControlFlowTok{if}\NormalTok{ (y }\OperatorTok{==}\StringTok{ }\DecValTok{0}\NormalTok{) \{}
  \ControlFlowTok{if}\NormalTok{ (x }\OperatorTok{>}\StringTok{ }\DecValTok{0}\NormalTok{) \{}
    \KeywordTok{log}\NormalTok{(x)}
\NormalTok{  \} }\ControlFlowTok{else}\NormalTok{ \{}
    \KeywordTok{message}\NormalTok{(}\StringTok{"x is negative or zero"}\NormalTok{)}
\NormalTok{  \}}
\NormalTok{\} }\ControlFlowTok{else}\NormalTok{ \{}
\NormalTok{  y }\OperatorTok{^}\StringTok{ }\NormalTok{x}
\NormalTok{\}}

\CommentTok{# Bad}
\ControlFlowTok{if}\NormalTok{ (y }\OperatorTok{<}\StringTok{ }\DecValTok{0} \OperatorTok{&&}\StringTok{ }\NormalTok{debug)}
\KeywordTok{message}\NormalTok{(}\StringTok{"Y is negative"}\NormalTok{)}

\ControlFlowTok{if}\NormalTok{ (y }\OperatorTok{==}\StringTok{ }\DecValTok{0}\NormalTok{)}
\NormalTok{\{}
    \ControlFlowTok{if}\NormalTok{ (x }\OperatorTok{>}\StringTok{ }\DecValTok{0}\NormalTok{) \{}
      \KeywordTok{log}\NormalTok{(x)}
\NormalTok{    \} }\ControlFlowTok{else}\NormalTok{ \{}
  \KeywordTok{message}\NormalTok{(}\StringTok{"x is negative or zero"}\NormalTok{)}
\NormalTok{    \}}
\NormalTok{\} }\ControlFlowTok{else}\NormalTok{ \{ y }\OperatorTok{^}\StringTok{ }\NormalTok{x \}}
\end{Highlighting}
\end{Shaded}

\subsection{Long lines}\label{long-lines}

Strive to limit your code to 80 characters per line. This fits
comfortably on a printed page with a reasonably sized font. If you find
yourself running out of room, this is a good indication that you should
encapsulate some of the work into a separate function.

If a function call is too long to fit on a single line, use one line
each for the function name, each argument, and the closing \texttt{)}.
This makes the code easier to read and to change later.

\begin{Shaded}
\begin{Highlighting}[]
\CommentTok{# Good}
\KeywordTok{do_something_very_complicated}\NormalTok{(}
  \DataTypeTok{something =} \StringTok{"that"}\NormalTok{,}
  \DataTypeTok{requires  =}\NormalTok{ many,}
  \DataTypeTok{arguments =} \StringTok{"some of which may be long"}
\NormalTok{)}

\CommentTok{# Bad}
\KeywordTok{do_something_very_complicated}\NormalTok{(}\StringTok{"that"}\NormalTok{, requires, many, arguments,}
                              \StringTok{"some of which may be long"}
\end{Highlighting}
\end{Shaded}

\subsection{Other}\label{other}

\begin{itemize}
\tightlist
\item
  Use \texttt{\textless{}-}, not \texttt{=}, for assignment. Keep
  \texttt{=} for parameters.
\end{itemize}

\begin{Shaded}
\begin{Highlighting}[]
\CommentTok{# Good}
\NormalTok{x <-}\StringTok{ }\DecValTok{5}
\KeywordTok{system.time}\NormalTok{(}
\NormalTok{  x <-}\StringTok{ }\KeywordTok{rnorm}\NormalTok{(}\FloatTok{1e6}\NormalTok{)}
\NormalTok{)}

\CommentTok{# Bad}
\NormalTok{x =}\StringTok{ }\DecValTok{5}
\KeywordTok{system.time}\NormalTok{(}
  \DataTypeTok{x =} \KeywordTok{rnorm}\NormalTok{(}\FloatTok{1e6}\NormalTok{)}
\NormalTok{)}
\end{Highlighting}
\end{Shaded}

\begin{itemize}
\item
  Don't put \texttt{;} at the end of a line, and don't use \texttt{;} to
  put multiple commands on one line.
\item
  Only use \texttt{return()} for early returns. Otherwise rely on R to
  return the result of the last evaluated expression.
\end{itemize}

\begin{Shaded}
\begin{Highlighting}[]
\CommentTok{# Good}
\NormalTok{add_two <-}\StringTok{ }\ControlFlowTok{function}\NormalTok{(x, y) \{}
\NormalTok{  x }\OperatorTok{+}\StringTok{ }\NormalTok{y}
\NormalTok{\}}

\CommentTok{# Bad}
\NormalTok{add_two <-}\StringTok{ }\ControlFlowTok{function}\NormalTok{(x, y) \{}
  \KeywordTok{return}\NormalTok{(x }\OperatorTok{+}\StringTok{ }\NormalTok{y)}
\NormalTok{\}}
\end{Highlighting}
\end{Shaded}

\begin{itemize}
\tightlist
\item
  Use \texttt{"}, not \texttt{\textquotesingle{}}, for quoting text. The
  only exception is when the text already contains double quotes and no
  single quotes.
\end{itemize}

\begin{Shaded}
\begin{Highlighting}[]
\CommentTok{# Good}
\StringTok{"Text"}
\StringTok{'Text with "quotes"'}
\StringTok{'<a href="http://style.tidyverse.org">A link</a>'}

\CommentTok{# Bad}
\StringTok{'Text'}
\StringTok{'Text with "double" and }\CharTok{\textbackslash{}'}\StringTok{single}\CharTok{\textbackslash{}'}\StringTok{ quotes'}
\end{Highlighting}
\end{Shaded}

\section{Coding practices}\label{coding-practices}

\subsection{Variables}\label{variables}

Create variables for values that are likely to change.

\subsection[\emph{Rule of Three}]{\texorpdfstring{\emph{Rule of
Three}\footnote{This is sometimes called the DRY principle, or Don't
  Repeat Yourself.}}{Rule of Three}}\label{rule-of-threedry}

Try not to copy code, or copy then modify the code, more than twice.

\begin{itemize}
\tightlist
\item
  If a change requires you to search/replace 3 or more times \emph{make
  a variable}.
\item
  If you copy a code chunk 3 or more times \emph{make a function}
\item
  If you copy a function 3 or more times \emph{make your function more
  generic}
\item
  If you copy a function into a project 3 or more times \emph{make a
  package}
\item
  If 3 or more people will use the function \emph{make a package}
\end{itemize}

The \emph{Rule of Three} applies to look-up tables and such also. The
key thing to think about is; if something changes how many touch points
will there be? If it is 3 or more places it is time to abstract this
code a bit.

\subsection{Path names}\label{path-names}

It is better to use relative path names instead of hard coded ones. If
you must read from (or write to) paths that are not in your project
directory structure create a file name variable at the highest level you
can (\emph{always end with the \texttt{/}}) and then use relative
paths.\\
\textbf{DO NOT EVER USE \texttt{setwd()}}

\begin{Shaded}
\begin{Highlighting}[]
\CommentTok{# Good}
\NormalTok{raw_data <-}\StringTok{ }\KeywordTok{read.csv}\NormalTok{(}\StringTok{"./data/mydatafile.csv"}\NormalTok{) }

\NormalTok{input_file <-}\StringTok{ "./data/mydatafile.csv"}
\NormalTok{raw_data <-}\StringTok{ }\KeywordTok{read.csv}\NormalTok{(input_file)  }

\NormalTok{input_path <-}\StringTok{ "C:/Path/To/Some/other/project/directory/"}
\NormalTok{input_file <-}\StringTok{ }\KeywordTok{paste0}\NormalTok{(input_path, }\StringTok{"data/mydatafile.csv"}\NormalTok{)}
\NormalTok{raw_data <-}\StringTok{ }\KeywordTok{read.csv}\NormalTok{(input_file)}

\CommentTok{# Bad}
\KeywordTok{setwd}\NormalTok{(}\StringTok{"C:/Path/To/Some/other/project/directory/data/"}\NormalTok{)}
\NormalTok{raw_data <-}\StringTok{ }\KeywordTok{read.csv}\NormalTok{(}\StringTok{"mydatafile.csv"}\NormalTok{)}
\KeywordTok{setwd}\NormalTok{(}\StringTok{"C:/Path/back/to/my/project/"}\NormalTok{)}
\end{Highlighting}
\end{Shaded}

\section{RStudio}\label{rstudio}

Download the latest version of
\href{https://www.rstudio.com/products/rstudio/download/\#download}{RStudio}
(\textgreater{} 1.1) and use it!

Learn more about new features of RStudio v1.1
\href{https://www.rstudio.com/resources/videos/rstudio-1-1-new-features/}{there}.

RStudio features:

\begin{itemize}
\item
  everything you can expect from a good IDE
\item
  keyboard shortcuts I use frequently

  \begin{enumerate}
  \def\labelenumi{\arabic{enumi}.}
  \tightlist
  \item
    \emph{Ctrl + Space} (auto-completion, better than \emph{Tab})
  \item
    \emph{Ctrl + Up} (command history \& search)
  \item
    \emph{Ctrl + Enter} (execute line of code)
  \item
    \emph{Ctrl + Shift + A} (reformat code)
  \item
    \emph{Ctrl + Shift + C} (comment/uncomment selected lines)
  \item
    \emph{Ctrl + Shift + /} (reflow comments)
  \item
    \emph{Ctrl + Shift + O} (View code outline)
  \item
    \emph{Ctrl + Shift + B} (build package, website or book)
  \item
    \emph{Ctrl + Shift + M} (pipe)
  \item
    \emph{Alt + Shift + K} to see all shortcuts\ldots{}
  \end{enumerate}
\item
  Panels (everything is integrated, including \textbf{Git} and a
  terminal)
\item
  Interactive data importation from files and connections (see
  \href{https://www.rstudio.com/resources/webinars/importing-data-into-r/}{this
  webinar})
\item
  Use
  \href{https://support.rstudio.com/hc/en-us/articles/205753617-Code-Diagnostics}{code
  diagnostics}:
\item
  \textbf{R Projects}:

  \begin{itemize}
  \tightlist
  \item
    \textbf{Meaningful structure} in one folder
  \item
    The working directory automatically switches to the project's folder
  \item
    File tab displays the associated files and folders in the project
  \item
    History of R commands and open files
  \item
    Any settings associated with the project, such as Git settings, are
    loaded. Note that a \emph{set-up.R} or even a \emph{.Rprofile} file
    in the project's root directory enable project-specific settings to
    be loaded each time people work on the project.
  \end{itemize}
\end{itemize}

The only two things that make @JennyBryan 😤😠🤯. Instead use projects +
here::here() \#rstats pic.twitter.com/GwxnHePL4n

--- Hadley Wickham (@hadleywickham) December 11 2017

Read more at
\url{https://www.tidyverse.org/articles/2017/12/workflow-vs-script/} and
also see chapter
\href{https://bookdown.org/csgillespie/efficientR/set-up.html}{\emph{Efficient
set-up}} of book \emph{Efficient R programming}.

\section{Getting help}\label{getting-help}

\subsection{Help yourself, learn how to
debug}\label{help-yourself-learn-how-to-debug}

A basic solution is to print everything, but it usually does not work
well on complex problems. A convenient solution to see all the
variables' states in your code is to place some \texttt{browser()}
anywhere you want to check the variables' states.

Learn more with
\href{https://bookdown.org/rdpeng/rprogdatascience/debugging.html}{this
book chapter},
\href{http://adv-r.had.co.nz/Exceptions-Debugging.html}{this other book
chapter},
\href{https://www.rstudio.com/resources/videos/debugging-techniques-in-rstudio/}{this
webinar} and
\href{https://support.rstudio.com/hc/en-us/articles/205612627-Debugging-with-RStudio}{this
RStudio article}.

\subsection{External help}\label{external-help}

Can't remember useful functions? Use
\href{https://www.rstudio.com/resources/cheatsheets/}{cheat sheets}.

You can search for specific R stuff on \url{https://rseek.org/}. You
should also read documentations carefully. If you're using a package,
search for vignettes and a GitHub repository.

You can also use \href{https://stackoverflow.com/}{Stack Overflow}. The
most common use of Stack Overflow is when you have an error or a
question, you Google it, and most of the times the first links are Q/A
on Stack Overflow.

You can ask questions on Stack Overflow (using the tag \texttt{r}). You
need to
\href{https://stackoverflow.com/questions/5963269/how-to-make-a-great-r-reproducible-example}{make
a great R reproducible example} if you want your question to be
answered. Most of the times, while making this reproducible example, you
will find the answer to your problem.

Join the \href{https://www.r-project.org/mail.html}{R-help} mailing
list. Sign up to get the daily digest and scan it for questions that
interest you.

\section{Keeping up to date}\label{keeping-up-to-date}

With over 10,000 packages on CRAN it is hard to keep up with the
constantly changing landscape.
\href{https://www.r-bloggers.com/}{R-Bloggers} is an R focused blog
aggregation site with dozens of posts per day. Check it out.

\section{Reading For Next Class}\label{reading-for-next-class}

\begin{enumerate}
\def\labelenumi{\arabic{enumi}.}
\tightlist
\item
  Read the chapter on
  \href{http://r4ds.had.co.nz/workflow-basics.html}{Workflow: basics}
\item
  Read the chapter on
  \href{http://r4ds.had.co.nz/workflow-scripts.html}{Workflow: scripts}
\item
  Read the chapter on
  \href{http://r4ds.had.co.nz/workflow-projects.html}{Workflow:
  projects}
\item
  Read Chapters 1-3 of the
  \href{http://style.tidyverse.org/index.html}{Tidyverse Style Guide}
\item
  See these RStudio
  \href{https://rviews.rstudio.com/categories/tips-and-tricks/}{Tips \&
  Tricks} or \href{https://twitter.com/rstudiotips}{these} and find one
  that looks interesting and \textbf{practice} it all week.
\item
  Read how to
  \href{https://stackoverflow.com/questions/5963269/how-to-make-a-great-r-reproducible-example}{make
  a great R reproducible example}
\end{enumerate}

\section{Exercises}\label{exercises}

\begin{enumerate}
\def\labelenumi{\arabic{enumi}.}
\tightlist
\item
  Create an R Project for this class.
\item
  Create the following directories in your project (tip sheet?)

  \begin{itemize}
  \tightlist
  \item
    Bonus points if you can do it from R and not RStudio or Windows
    Explorer
  \item
    Double Bonus points if you can make it a function.
  \item
    Hint: In the R console type \texttt{file} and scroll through the
    various functions which appear in the pop-up.
  \end{itemize}
\item
  Copy one of your R scripts into your R directory. (Bonus points if you
  can do it from R and not RStudio or Windows Explorer)
\item
  Apply the style guide to your code.\\
\item
  Apply the ``Rule of 3''

  \begin{itemize}
  \tightlist
  \item
    Create variables as needed
  \item
    Identify code that is used 3 or more times to make functions
  \item
    Identify code that would be useful in 3 or more projects to
    integrate into a package.
  \end{itemize}
\end{enumerate}

\part{Base R Basics}\label{part-base-r-basics}

\chapter{R Basics}\label{baser-rbasics}

Here is a quick overview of the basics. Next we'll dive deep into R's
basic data structures and then how to subset these data structures. This
will give us a good overview of base R and the background needed to dive
into \textbf{R for Data Science}.

The three most important functions in R \texttt{?}, \texttt{??}, and
\texttt{str}:

\begin{itemize}
\tightlist
\item
  \texttt{?topic} provides access to the documentation for \emph{topic}.
\item
  \texttt{??topic} searches the documentation for \emph{topic}.
\item
  \texttt{str} displays the structure of an R object in human readable
  form.
\end{itemize}

See this \href{http://adv-r.had.co.nz/Vocabulary.html}{vocabulary list}
for a good starting point on the basics functions in base R and some
important libraries.

A book to learn the basics is
\href{https://bookdown.org/rdpeng/rprogdatascience/}{R Programming for
Data Science}

In R there three basic constructs\footnote{Technically speaking
  functions and environments are objects which allows one to do things
  in R you can't do in many other languages.}; objects, functions, and
environments.

\section{Assignment Operators}\label{assignment-operators}

We saw this is Coding Style. Use \texttt{\textless{}-} for assignment
and use \texttt{=} for parameters. While you can use \texttt{=} for
assignment it is generally considered bad practice.

\section{Naming Rules}\label{naming-rules}

R has strict rules about what constitutes a valid name. A
\textbf{syntactic} name must consist of letters\footnote{Surprisingly,
  what constitutes a letter is determined by your current locale. That
  means that the syntax of R code actually differs from computer to
  computer, and it's possible for a file that works on one computer to
  not even parse on another!}, digits, \texttt{.} and \texttt{\_}, and
can't begin with \texttt{\_}. Additionally, it can not be one of a list
of \textbf{reserved words} like \texttt{TRUE}, \texttt{NULL},
\texttt{if}, and \texttt{function} (see the complete list in
\texttt{?Reserved}). Names that don't follow these rules are called
\textbf{non-syntactic} names, and if you try to use them, you'll get an
error:

\begin{Shaded}
\begin{Highlighting}[]
\NormalTok{_abc <-}\StringTok{ }\DecValTok{1}
\CommentTok{#> Error: unexpected input in "_"}

\ControlFlowTok{if}\NormalTok{ <-}\StringTok{ }\DecValTok{10}
\CommentTok{#> Error: unexpected assignment in "if <-"}
\end{Highlighting}
\end{Shaded}

\begin{rmdwarning}
While \texttt{TRUE} and \texttt{FALSE} are reserved words \texttt{T} and
\texttt{F} are not. However, you can use \texttt{T} and \texttt{F} as
logical. If someone assigns either of those a different value you will
get a \textbf{very} hard to track down bug. Always spell out the
\texttt{TRUE} and \texttt{FALSE}.
\end{rmdwarning}

\section{Objects}\label{objects}

\subsection{Vector}\label{vector}

You create a vector with \texttt{c}. These have to be the same data type
(See next section).

\begin{Shaded}
\begin{Highlighting}[]
\NormalTok{v <-}\StringTok{ }\KeywordTok{c}\NormalTok{(}\StringTok{"my"}\NormalTok{, }\StringTok{"first"}\NormalTok{, }\StringTok{"vector"}\NormalTok{)}
\NormalTok{v}
\CommentTok{#> [1] "my"     "first"  "vector"}

\CommentTok{# length of our vector}
\KeywordTok{length}\NormalTok{(v)}
\CommentTok{#> [1] 3}
\end{Highlighting}
\end{Shaded}

There are several shortcut functions for common vector creation.

\begin{Shaded}
\begin{Highlighting}[]
\CommentTok{# create an ordered sequence}
\DecValTok{2}\OperatorTok{:}\DecValTok{10}
\CommentTok{#> [1]  2  3  4  5  6  7  8  9 10}
\DecValTok{9}\OperatorTok{:}\DecValTok{3}
\CommentTok{#> [1] 9 8 7 6 5 4 3}

\CommentTok{# generate regular sequences}
\KeywordTok{seq}\NormalTok{(}\DecValTok{1}\NormalTok{, }\DecValTok{20}\NormalTok{, }\DataTypeTok{by =} \DecValTok{3}\NormalTok{)}
\CommentTok{#> [1]  1  4  7 10 13 16 19}

\CommentTok{# replicate a number n times}
\KeywordTok{rep}\NormalTok{(}\DecValTok{3}\NormalTok{, }\DataTypeTok{times =} \DecValTok{4}\NormalTok{)}
\CommentTok{#> [1] 3 3 3 3}

\CommentTok{# arguments are generally vectorized}
\KeywordTok{rep}\NormalTok{(}\DecValTok{1}\OperatorTok{:}\DecValTok{3}\NormalTok{, }\DataTypeTok{times =} \DecValTok{3}\OperatorTok{:}\DecValTok{1}\NormalTok{)}
\CommentTok{#> [1] 1 1 1 2 2 3}

\CommentTok{# common mistake using 1:length(n) in loops}
\CommentTok{# but if n = 0}
\DecValTok{1}\OperatorTok{:}\DecValTok{0}
\CommentTok{#> [1] 1 0}

\CommentTok{# use seq_len(n) instead and the loop won't execute}
\KeywordTok{seq_len}\NormalTok{(}\DecValTok{0}\NormalTok{)}
\CommentTok{#> integer(0)}

\CommentTok{# another common mistake}
\NormalTok{n <-}\StringTok{ }\DecValTok{6}
\DecValTok{1}\OperatorTok{:}\NormalTok{n}\OperatorTok{+}\DecValTok{1}        \CommentTok{# is (1:n) + 1, so 2:(n + 1)}
\CommentTok{#> [1] 2 3 4 5 6 7}
\DecValTok{1}\OperatorTok{:}\NormalTok{(n}\OperatorTok{+}\DecValTok{1}\NormalTok{)      }\CommentTok{# usually what is meant}
\CommentTok{#> [1] 1 2 3 4 5 6 7}
\KeywordTok{seq_len}\NormalTok{(n}\OperatorTok{+}\DecValTok{1}\NormalTok{) }\CommentTok{# a better way}
\CommentTok{#> [1] 1 2 3 4 5 6 7}
\end{Highlighting}
\end{Shaded}

\subsection{Atomic Vectors}\label{atomic-vectors}

There are many ``atomic'' types of data: \texttt{logical},
\texttt{integer}, \texttt{double} and \texttt{character} (in this order,
see below). There are also \texttt{raw} and \texttt{complex} but they
are rarely used.

You can't mix types in an atomic vector (you can in a list). Coercion
will automatically occur if you mix types:

\begin{Shaded}
\begin{Highlighting}[]
\NormalTok{(a <-}\StringTok{ }\OtherTok{FALSE}\NormalTok{)}
\CommentTok{#> [1] FALSE}
\KeywordTok{typeof}\NormalTok{(a)}
\CommentTok{#> [1] "logical"}

\NormalTok{(b <-}\StringTok{ }\DecValTok{1}\OperatorTok{:}\DecValTok{10}\NormalTok{)}
\CommentTok{#>  [1]  1  2  3  4  5  6  7  8  9 10}
\KeywordTok{typeof}\NormalTok{(b)}
\CommentTok{#> [1] "integer"}
\KeywordTok{c}\NormalTok{(a, b)         ## FALSE is coerced to integer 0}
\CommentTok{#>  [1]  0  1  2  3  4  5  6  7  8  9 10}

\NormalTok{(c <-}\StringTok{ }\FloatTok{10.5}\NormalTok{)}
\CommentTok{#> [1] 10.5}
\KeywordTok{typeof}\NormalTok{(c)}
\CommentTok{#> [1] "double"}
\NormalTok{(d <-}\StringTok{ }\KeywordTok{c}\NormalTok{(b, c))  ## coerced to double}
\CommentTok{#>  [1]  1.0  2.0  3.0  4.0  5.0  6.0  7.0  8.0  9.0 10.0 10.5}

\KeywordTok{c}\NormalTok{(d, }\StringTok{"a"}\NormalTok{)       ## coerced to character}
\CommentTok{#>  [1] "1"    "2"    "3"    "4"    "5"    "6"    "7"    "8"    "9"    "10"  }
\CommentTok{#> [11] "10.5" "a"}

\DecValTok{50} \OperatorTok{<}\StringTok{ "7"}
\CommentTok{#> [1] TRUE}
\end{Highlighting}
\end{Shaded}

You can force coercion with \texttt{as.logical}, \texttt{as.integer},
\texttt{as.double}, \texttt{as.numeric}, and \texttt{as.character}. Most
of the time the coercion rules are straight forward, but not always.

\begin{Shaded}
\begin{Highlighting}[]
\NormalTok{x <-}\StringTok{ }\KeywordTok{c}\NormalTok{(}\OtherTok{TRUE}\NormalTok{, }\OtherTok{FALSE}\NormalTok{)}
\KeywordTok{typeof}\NormalTok{(x)}
\CommentTok{#> [1] "logical"}

\KeywordTok{as.integer}\NormalTok{(x)}
\CommentTok{#> [1] 1 0}
\KeywordTok{as.numeric}\NormalTok{(x)}
\CommentTok{#> [1] 1 0}
\KeywordTok{as.character}\NormalTok{(x)}
\CommentTok{#> [1] "TRUE"  "FALSE"}
\end{Highlighting}
\end{Shaded}

However, coercion is not associative.

\begin{Shaded}
\begin{Highlighting}[]
\NormalTok{x <-}\StringTok{ }\KeywordTok{c}\NormalTok{(}\OtherTok{TRUE}\NormalTok{, }\OtherTok{FALSE}\NormalTok{)}

\NormalTok{x2 <-}\StringTok{ }\KeywordTok{as.integer}\NormalTok{(x)}
\NormalTok{x3 <-}\StringTok{ }\KeywordTok{as.numeric}\NormalTok{(x2)}
\KeywordTok{as.character}\NormalTok{(x3)}
\CommentTok{#> [1] "1" "0"}
\end{Highlighting}
\end{Shaded}

What would you expect this to return?

\begin{Shaded}
\begin{Highlighting}[]
\NormalTok{x <-}\StringTok{ }\KeywordTok{c}\NormalTok{(}\OtherTok{TRUE}\NormalTok{, }\OtherTok{FALSE}\NormalTok{)}

\KeywordTok{as.integer}\NormalTok{(}\KeywordTok{as.character}\NormalTok{(x))}
\end{Highlighting}
\end{Shaded}

You can test for an ``atomic'' types of data with: \texttt{is.logical},
\texttt{is.integer}, \texttt{is.double}, \texttt{is.numeric}\footnote{\texttt{is.numeric()}
  is a general test for the ``numberliness'' of a vector and returns
  TRUE for both integer and double vectors. It is not a specific test
  for double vectors, which are often called numeric.}, and
\texttt{is.character}.

\begin{Shaded}
\begin{Highlighting}[]
\NormalTok{x <-}\StringTok{ }\KeywordTok{c}\NormalTok{(}\OtherTok{TRUE}\NormalTok{, }\OtherTok{FALSE}\NormalTok{)}

\KeywordTok{is.logical}\NormalTok{(x)}
\CommentTok{#> [1] TRUE}
\KeywordTok{is.integer}\NormalTok{(x)}
\CommentTok{#> [1] FALSE}
\end{Highlighting}
\end{Shaded}

What would you expect these to return?

\begin{Shaded}
\begin{Highlighting}[]
\NormalTok{x <-}\StringTok{ }\DecValTok{2}

\KeywordTok{is.integer}\NormalTok{(x)}
\KeywordTok{is.numeric}\NormalTok{(x)}
\KeywordTok{is.double}\NormalTok{(x)}
\end{Highlighting}
\end{Shaded}

Missing values are specified with \texttt{NA}, which is a logical vector
of length 1. \texttt{NA} will always be coerced to the correct type if
used inside \texttt{c()}, or you can create \texttt{NA}s of a specific
type with \texttt{NA\_real\_} (a double vector), \texttt{NA\_integer\_}
and \texttt{NA\_character\_}.

\subsection{Matrix}\label{matrix}

Matrices are 2D vectors, with all elements of the same type. Generally
used for mathematics.

\begin{Shaded}
\begin{Highlighting}[]
\CommentTok{# fill in column order (default)}
\KeywordTok{matrix}\NormalTok{(}\DecValTok{1}\OperatorTok{:}\DecValTok{12}\NormalTok{, }\DataTypeTok{nrow =} \DecValTok{3}\NormalTok{)}
\CommentTok{#>      [,1] [,2] [,3] [,4]}
\CommentTok{#> [1,]    1    4    7   10}
\CommentTok{#> [2,]    2    5    8   11}
\CommentTok{#> [3,]    3    6    9   12}

\CommentTok{# fill in row order}
\KeywordTok{matrix}\NormalTok{(}\DecValTok{1}\OperatorTok{:}\DecValTok{12}\NormalTok{, }\DataTypeTok{nrow =} \DecValTok{3}\NormalTok{, }\DataTypeTok{byrow =} \OtherTok{TRUE}\NormalTok{)}
\CommentTok{#>      [,1] [,2] [,3] [,4]}
\CommentTok{#> [1,]    1    2    3    4}
\CommentTok{#> [2,]    5    6    7    8}
\CommentTok{#> [3,]    9   10   11   12}

\CommentTok{# can also specify the number of columns instead}
\KeywordTok{matrix}\NormalTok{(}\DecValTok{1}\OperatorTok{:}\DecValTok{12}\NormalTok{, }\DataTypeTok{ncol =} \DecValTok{3}\NormalTok{)}
\CommentTok{#>      [,1] [,2] [,3]}
\CommentTok{#> [1,]    1    5    9}
\CommentTok{#> [2,]    2    6   10}
\CommentTok{#> [3,]    3    7   11}
\CommentTok{#> [4,]    4    8   12}
\end{Highlighting}
\end{Shaded}

You find the dimensions of a matrix with \texttt{nrow}, \texttt{ncol},
and \texttt{dim}

\begin{Shaded}
\begin{Highlighting}[]
\NormalTok{m <-}\StringTok{ }\KeywordTok{matrix}\NormalTok{(}\DecValTok{1}\OperatorTok{:}\DecValTok{12}\NormalTok{, }\DataTypeTok{ncol =} \DecValTok{3}\NormalTok{)}
\KeywordTok{dim}\NormalTok{(m)}
\CommentTok{#> [1] 4 3}
\KeywordTok{nrow}\NormalTok{(m)}
\CommentTok{#> [1] 4}
\KeywordTok{ncol}\NormalTok{(m)}
\CommentTok{#> [1] 3}
\end{Highlighting}
\end{Shaded}

\subsection{List}\label{list}

A list is a generic vector containing other objects. These do
\textbf{NOT} have to be the same type or the same length.

\begin{Shaded}
\begin{Highlighting}[]
\NormalTok{s <-}\StringTok{ }\KeywordTok{c}\NormalTok{(}\StringTok{"aa"}\NormalTok{, }\StringTok{"bb"}\NormalTok{, }\StringTok{"cc"}\NormalTok{, }\StringTok{"dd"}\NormalTok{, }\StringTok{"ee"}\NormalTok{) }
\NormalTok{b <-}\StringTok{ }\KeywordTok{c}\NormalTok{(}\OtherTok{TRUE}\NormalTok{, }\OtherTok{FALSE}\NormalTok{, }\OtherTok{TRUE}\NormalTok{, }\OtherTok{FALSE}\NormalTok{, }\OtherTok{FALSE}\NormalTok{) }
\CommentTok{# x contains copies of n, s, b and our matrix from above}
\NormalTok{x <-}\StringTok{ }\KeywordTok{list}\NormalTok{(}\DataTypeTok{n =} \KeywordTok{c}\NormalTok{(}\DecValTok{2}\NormalTok{, }\DecValTok{3}\NormalTok{, }\DecValTok{5}\NormalTok{) , s, b, }\DecValTok{3}\NormalTok{, m)   }
\NormalTok{x}
\CommentTok{#> $n}
\CommentTok{#> [1] 2 3 5}
\CommentTok{#> }
\CommentTok{#> [[2]]}
\CommentTok{#> [1] "aa" "bb" "cc" "dd" "ee"}
\CommentTok{#> }
\CommentTok{#> [[3]]}
\CommentTok{#> [1]  TRUE FALSE  TRUE FALSE FALSE}
\CommentTok{#> }
\CommentTok{#> [[4]]}
\CommentTok{#> [1] 3}
\CommentTok{#> }
\CommentTok{#> [[5]]}
\CommentTok{#>      [,1] [,2] [,3]}
\CommentTok{#> [1,]    1    5    9}
\CommentTok{#> [2,]    2    6   10}
\CommentTok{#> [3,]    3    7   11}
\CommentTok{#> [4,]    4    8   12}

\CommentTok{# length gives you length of the list not the elements in the list}
\KeywordTok{length}\NormalTok{(x)}
\CommentTok{#> [1] 5}
\end{Highlighting}
\end{Shaded}

We'll discuss lists in more detail later in the course.

\subsection{Data frame}\label{data-frame}

A data frame is a list with each vector of the same length. This is the
main data structure used and is analogous to a data set in SAS. While
these \textbf{look} like matrices they behave very different.

\begin{Shaded}
\begin{Highlighting}[]
\NormalTok{df =}\StringTok{ }\KeywordTok{data.frame}\NormalTok{(}\DataTypeTok{n =} \KeywordTok{c}\NormalTok{(}\DecValTok{2}\NormalTok{, }\DecValTok{3}\NormalTok{, }\DecValTok{5}\NormalTok{),}
                \DataTypeTok{s =} \KeywordTok{c}\NormalTok{(}\StringTok{"aa"}\NormalTok{, }\StringTok{"bb"}\NormalTok{, }\StringTok{"cc"}\NormalTok{),}
                \DataTypeTok{b =} \KeywordTok{c}\NormalTok{(}\OtherTok{TRUE}\NormalTok{, }\OtherTok{FALSE}\NormalTok{, }\OtherTok{TRUE}\NormalTok{),}
                \DataTypeTok{y =}\NormalTok{ v}
\NormalTok{                )       }\CommentTok{# df is a data frame }
\NormalTok{df}
\CommentTok{#>   n  s     b      y}
\CommentTok{#> 1 2 aa  TRUE     my}
\CommentTok{#> 2 3 bb FALSE  first}
\CommentTok{#> 3 5 cc  TRUE vector}

\CommentTok{# dimensions}
\KeywordTok{dim}\NormalTok{(df)}
\CommentTok{#> [1] 3 4}
\KeywordTok{nrow}\NormalTok{(df)}
\CommentTok{#> [1] 3}
\KeywordTok{ncol}\NormalTok{(df)}
\CommentTok{#> [1] 4}
\KeywordTok{length}\NormalTok{(df)}
\CommentTok{#> [1] 4}
\end{Highlighting}
\end{Shaded}

We'll discuss data frames in greater detail later in the course.

\section{Comparision}\label{comparision}

\begin{table}

\caption{\label{tab:table-logicalOps}Logical Operators}
\centering
\begin{tabular}[t]{ll}
\toprule
Operator & Description\\
\midrule
> & greater than\\
>= & greater than or equal to\\
< & less than\\
<= & less than or equal to\\
== & exactly equal to\\
!= & not equal to\\
\bottomrule
\end{tabular}
\end{table}

\begin{Shaded}
\begin{Highlighting}[]
\NormalTok{v <-}\StringTok{ }\DecValTok{1}\OperatorTok{:}\DecValTok{12}
\NormalTok{v[v }\OperatorTok{>}\StringTok{ }\DecValTok{9}\NormalTok{]}
\CommentTok{#> [1] 10 11 12}
\end{Highlighting}
\end{Shaded}

Equality can be tricky to test for since real numbers can't be expressed
exactly in computers.

\begin{Shaded}
\begin{Highlighting}[]
\NormalTok{x <-}\StringTok{ }\KeywordTok{sqrt}\NormalTok{(}\DecValTok{2}\NormalTok{)}
\NormalTok{(y <-}\StringTok{ }\NormalTok{x}\OperatorTok{^}\DecValTok{2}\NormalTok{)}
\CommentTok{#> [1] 2}
\NormalTok{y }\OperatorTok{==}\StringTok{ }\DecValTok{2}
\CommentTok{#> [1] FALSE}
\KeywordTok{print}\NormalTok{(y, }\DataTypeTok{digits =} \DecValTok{20}\NormalTok{)}
\CommentTok{#> [1] 2.0000000000000004}
\KeywordTok{all.equal}\NormalTok{(y, }\DecValTok{2}\NormalTok{)          ## equality with some tolerance}
\CommentTok{#> [1] TRUE}
\KeywordTok{all.equal}\NormalTok{(y, }\DecValTok{3}\NormalTok{)}
\CommentTok{#> [1] "Mean relative difference: 0.5"}
\KeywordTok{isTRUE}\NormalTok{(}\KeywordTok{all.equal}\NormalTok{(y, }\DecValTok{3}\NormalTok{))  ## if you want a boolean, use isTRUE()}
\CommentTok{#> [1] FALSE}
\end{Highlighting}
\end{Shaded}

\section{Logical and sets}\label{logical-and-sets}

\begin{Shaded}
\begin{Highlighting}[]
\NormalTok{x <-}\StringTok{ }\KeywordTok{c}\NormalTok{(}\OtherTok{TRUE}\NormalTok{, }\OtherTok{FALSE}\NormalTok{)}
\NormalTok{df <-}\StringTok{ }\KeywordTok{data.frame}\NormalTok{(}\KeywordTok{expand.grid}\NormalTok{(x, x))}
\KeywordTok{names}\NormalTok{(df) <-}\StringTok{ }\KeywordTok{c}\NormalTok{(}\StringTok{"x"}\NormalTok{, }\StringTok{"y"}\NormalTok{)}
\NormalTok{df}\OperatorTok{$}\NormalTok{and  <-}\StringTok{ }\NormalTok{df}\OperatorTok{$}\NormalTok{x }\OperatorTok{&}\StringTok{ }\NormalTok{df}\OperatorTok{$}\NormalTok{y     }\CommentTok{# logical and}
\NormalTok{df}\OperatorTok{$}\NormalTok{or   <-}\StringTok{ }\NormalTok{df}\OperatorTok{$}\NormalTok{x }\OperatorTok{|}\StringTok{ }\NormalTok{df}\OperatorTok{$}\NormalTok{y     }\CommentTok{# logical or}
\NormalTok{df}\OperatorTok{$}\NormalTok{notx <-}\StringTok{ }\OperatorTok{!}\NormalTok{df}\OperatorTok{$}\NormalTok{x           }\CommentTok{# negation}
\NormalTok{df}\OperatorTok{$}\NormalTok{xor  <-}\StringTok{ }\KeywordTok{xor}\NormalTok{(df}\OperatorTok{$}\NormalTok{x, df}\OperatorTok{$}\NormalTok{y) }\CommentTok{# exlusive or}
\NormalTok{df}
\CommentTok{#>       x     y   and    or  notx   xor}
\CommentTok{#> 1  TRUE  TRUE  TRUE  TRUE FALSE FALSE}
\CommentTok{#> 2 FALSE  TRUE FALSE  TRUE  TRUE  TRUE}
\CommentTok{#> 3  TRUE FALSE FALSE  TRUE FALSE  TRUE}
\CommentTok{#> 4 FALSE FALSE FALSE FALSE  TRUE FALSE}
\end{Highlighting}
\end{Shaded}

R has two versions of the logical operators \texttt{\&} and
\texttt{\&\&} (\texttt{\textbar{}} and \texttt{\textbar{}\textbar{}}).
The single version is the vectorized version while the the double
version returns a length-one vector. Use the double version in logical
control structures (if, for, while, etc).

\begin{Shaded}
\begin{Highlighting}[]
\NormalTok{df}\OperatorTok{$}\NormalTok{x }\OperatorTok{&&}\StringTok{ }\NormalTok{df}\OperatorTok{$}\NormalTok{y  }\CommentTok{# only and the first elements}
\CommentTok{#> [1] TRUE}
\NormalTok{df}\OperatorTok{$}\NormalTok{x }\OperatorTok{||}\StringTok{ }\NormalTok{df}\OperatorTok{$}\NormalTok{y  }\CommentTok{# only or the first elements}
\CommentTok{#> [1] TRUE}
\end{Highlighting}
\end{Shaded}

This is a common source of bugs in control structures (if, for, while,
etc) where you must have a single TRUE / FALSE.

\begin{rmdcaution}
\texttt{=} is used for assignment while \texttt{==} is used for
comparison. A common bug is to use \texttt{=} instead of \texttt{==}
inside a control structure.
\end{rmdcaution}

It also has useful helpers \texttt{any} and \texttt{all}

\begin{Shaded}
\begin{Highlighting}[]
\NormalTok{x <-}\StringTok{ }\KeywordTok{c}\NormalTok{(}\OtherTok{FALSE}\NormalTok{, }\OtherTok{FALSE}\NormalTok{, }\OtherTok{FALSE}\NormalTok{, }\OtherTok{TRUE}\NormalTok{)}
\KeywordTok{any}\NormalTok{(x)}
\CommentTok{#> [1] TRUE}
\KeywordTok{all}\NormalTok{(x)}
\CommentTok{#> [1] FALSE}
\KeywordTok{all}\NormalTok{(}\OperatorTok{!}\NormalTok{x[}\DecValTok{1}\OperatorTok{:}\DecValTok{3}\NormalTok{])}
\CommentTok{#> [1] TRUE}
\end{Highlighting}
\end{Shaded}

And also some useful \textbf{set} operations \texttt{intersect},
\texttt{union}, \texttt{setdiff}, \texttt{setequal}

\begin{Shaded}
\begin{Highlighting}[]
\NormalTok{x <-}\StringTok{ }\DecValTok{1}\OperatorTok{:}\DecValTok{5}
\NormalTok{y <-}\StringTok{ }\DecValTok{3}\OperatorTok{:}\DecValTok{7}

\KeywordTok{intersect}\NormalTok{(x, y) }\CommentTok{# in x and in y}
\CommentTok{#> [1] 3 4 5}
\KeywordTok{union}\NormalTok{(x, y)     }\CommentTok{# different than c()}
\CommentTok{#> [1] 1 2 3 4 5 6 7}
\KeywordTok{c}\NormalTok{(x,y)          }\CommentTok{# not a set operation}
\CommentTok{#>  [1] 1 2 3 4 5 3 4 5 6 7}
\KeywordTok{setdiff}\NormalTok{(x, y)   }\CommentTok{# in x but not in y}
\CommentTok{#> [1] 1 2}
\KeywordTok{setdiff}\NormalTok{(y, x)   }\CommentTok{# in y but not in x}
\CommentTok{#> [1] 6 7}
\KeywordTok{setequal}\NormalTok{(x, y)}
\CommentTok{#> [1] FALSE}
\NormalTok{z <-}\StringTok{ }\DecValTok{5}\OperatorTok{:}\DecValTok{1}
\KeywordTok{setequal}\NormalTok{(x, z)}
\CommentTok{#> [1] TRUE}
\end{Highlighting}
\end{Shaded}

\section{Control Structures}\label{control-structures}

Control structures allow you to put some ``logic'' into your R code,
rather than just always executing the same R code every time. Control
structures allow you to respond to inputs or to features of the data and
execute different R expressions accordingly.

Commonly used control structures are

\begin{itemize}
\item
  \texttt{if} and \texttt{else}: testing a condition and acting on it
\item
  \texttt{for}: execute a loop a fixed number of times
\item
  \texttt{while}: execute a loop \emph{while} a condition is true
\item
  \texttt{repeat}: execute an infinite loop (must \texttt{break} out of
  it to stop)
\item
  \texttt{break}: break the execution of a loop
\item
  \texttt{next}: skip an iteration of a loop
\end{itemize}

\subsection{\texorpdfstring{\texttt{if}-\texttt{else}}{if-else}}\label{if-else}

The \texttt{if}-\texttt{else} combination is probably the most commonly
used control structure in R (or perhaps any language). This structure
allows you to test a condition and act on it depending on whether it's
true or false.

For starters, you can just use the \texttt{if} statement.

\begin{Shaded}
\begin{Highlighting}[]
\ControlFlowTok{if}\NormalTok{(}\OperatorTok{<}\NormalTok{condition}\OperatorTok{>}\NormalTok{) \{}
        \CommentTok{# do something}
\NormalTok{\} }
\CommentTok{# Continue with rest of code}
\end{Highlighting}
\end{Shaded}

The above code does nothing if the condition is false. If you have an
action you want to execute when the condition is false, then you need an
\texttt{else} clause.

\begin{Shaded}
\begin{Highlighting}[]
\ControlFlowTok{if}\NormalTok{(}\OperatorTok{<}\NormalTok{condition}\OperatorTok{>}\NormalTok{) \{}
        \CommentTok{# do something}
\NormalTok{\} }
\ControlFlowTok{else}\NormalTok{ \{}
        \CommentTok{# do something else}
\NormalTok{\}}
\end{Highlighting}
\end{Shaded}

You can have a series of tests by following the initial \texttt{if} with
any number of \texttt{else\ if}s.

\begin{Shaded}
\begin{Highlighting}[]
\ControlFlowTok{if}\NormalTok{(}\OperatorTok{<}\NormalTok{condition1}\OperatorTok{>}\NormalTok{) \{}
        \CommentTok{# do something}
\NormalTok{\} }\ControlFlowTok{else} \ControlFlowTok{if}\NormalTok{(}\OperatorTok{<}\NormalTok{condition2}\OperatorTok{>}\NormalTok{)  \{}
        \CommentTok{# do something different}
\NormalTok{\} }\ControlFlowTok{else}\NormalTok{ \{}
        \CommentTok{# do something else different}
\NormalTok{\}}
\end{Highlighting}
\end{Shaded}

\begin{rmdwarning}
There is also an \texttt{ifelse} function which is vectorized version.
It is essentially an \texttt{if}-\texttt{else} wrapped in a \texttt{for}
loop so that the condition, and action, is performed on each element in
a vector.
\end{rmdwarning}

\subsection{\texorpdfstring{\texttt{for}
Loops}{for Loops}}\label{for-loops}

For loops are pretty much the only looping construct that you will need
in R. While you may occasionally find a need for other types of loops,
in my experience doing data analysis, I've found very few situations
where a for loop wasn't sufficient.

In R, for loops take an iterator variable and assign it successive
values from a sequence or vector. For loops are most commonly used for
iterating over the elements of an object (list, vector, etc.)

The following three loops all have the similar behavior.

\begin{Shaded}
\begin{Highlighting}[]
\NormalTok{x <-}\StringTok{ }\KeywordTok{c}\NormalTok{(}\StringTok{"a"}\NormalTok{, }\StringTok{"b"}\NormalTok{, }\StringTok{"c"}\NormalTok{, }\StringTok{"d"}\NormalTok{)}

\ControlFlowTok{for}\NormalTok{(i }\ControlFlowTok{in} \DecValTok{1}\OperatorTok{:}\KeywordTok{length}\NormalTok{(x)) \{}
\NormalTok{        ## Print out each element of 'x'}
        \KeywordTok{print}\NormalTok{(x[i])  }
\NormalTok{\}}
\CommentTok{#> [1] "a"}
\CommentTok{#> [1] "b"}
\CommentTok{#> [1] "c"}
\CommentTok{#> [1] "d"}
\end{Highlighting}
\end{Shaded}

The \texttt{seq\_along()} function is commonly used in conjunction with
for loops in order to generate an integer sequence based on the length
of an object (in this case, the object \texttt{x}).

\begin{Shaded}
\begin{Highlighting}[]
\NormalTok{## Generate a sequence based on length of 'x'}
\ControlFlowTok{for}\NormalTok{(i }\ControlFlowTok{in} \KeywordTok{seq_along}\NormalTok{(x)) \{   }
        \KeywordTok{print}\NormalTok{(x[i])}
\NormalTok{\}}
\CommentTok{#> [1] "a"}
\CommentTok{#> [1] "b"}
\CommentTok{#> [1] "c"}
\CommentTok{#> [1] "d"}
\end{Highlighting}
\end{Shaded}

It is not necessary to use an index-type variable.

\begin{Shaded}
\begin{Highlighting}[]
\ControlFlowTok{for}\NormalTok{(letter }\ControlFlowTok{in}\NormalTok{ x) \{}
        \KeywordTok{print}\NormalTok{(letter)}
\NormalTok{\}}
\CommentTok{#> [1] "a"}
\CommentTok{#> [1] "b"}
\CommentTok{#> [1] "c"}
\CommentTok{#> [1] "d"}
\end{Highlighting}
\end{Shaded}

\BeginKnitrBlock{rmdtip}
Nested loops are commonly needed for multidimensional or hierarchical
data structures (e.g.~matrices, lists). Be careful with nesting though.
Nesting beyond 2 to 3 levels often makes it difficult to read/understand
the code. If you find yourself in need of a large number of nested
loops, you probably want to break up the loops by using functions
(discussed later).
\EndKnitrBlock{rmdtip}

We will discus looping and the other control structures in more detail
when we get to the section on iterators.

\section{Vectorization \& Recycling}\label{vectorization-recycling}

Many operations in R are \emph{vectorized}, meaning that operations
occur in parallel in certain R objects. This allows you to write code
that is efficient, concise, and easier to read than in non-vectorized
languages.

The simplest example is when adding two vectors together.

\begin{Shaded}
\begin{Highlighting}[]
\NormalTok{x <-}\StringTok{ }\DecValTok{1}\OperatorTok{:}\DecValTok{3}
\NormalTok{y <-}\StringTok{ }\DecValTok{11}\OperatorTok{:}\DecValTok{13}
\NormalTok{z <-}\StringTok{ }\NormalTok{x }\OperatorTok{+}\StringTok{ }\NormalTok{y}
\NormalTok{z}
\CommentTok{#> [1] 12 14 16}
\end{Highlighting}
\end{Shaded}

In most other languages you would have to do something like

\begin{Shaded}
\begin{Highlighting}[]
\NormalTok{z <-}\StringTok{ }\KeywordTok{numeric}\NormalTok{(}\KeywordTok{length}\NormalTok{(x))}

\ControlFlowTok{for}\NormalTok{(i }\ControlFlowTok{in} \KeywordTok{seq_along}\NormalTok{(x)) \{}
\NormalTok{      z[i] <-}\StringTok{ }\NormalTok{x[i] }\OperatorTok{+}\StringTok{ }\NormalTok{y[i]}
\NormalTok{\}}
\NormalTok{z}
\CommentTok{#> [1] 12 14 16}
\end{Highlighting}
\end{Shaded}

We saw a form of vectorization above in the logical operators.

\begin{Shaded}
\begin{Highlighting}[]
\NormalTok{x}
\CommentTok{#> [1] 1 2 3}
\NormalTok{x }\OperatorTok{>}\StringTok{ }\DecValTok{2}
\CommentTok{#> [1] FALSE FALSE  TRUE}
\NormalTok{x[x }\OperatorTok{>}\StringTok{ }\DecValTok{2}\NormalTok{]}
\CommentTok{#> [1] 3}
\end{Highlighting}
\end{Shaded}

Matrix operations are also vectorized, making for nice compact notation.
This way, we can do element-by-element operations on matrices without
having to loop over every element.

\begin{Shaded}
\begin{Highlighting}[]
\NormalTok{x <-}\StringTok{ }\KeywordTok{matrix}\NormalTok{(}\DecValTok{1}\OperatorTok{:}\DecValTok{4}\NormalTok{, }\DecValTok{2}\NormalTok{, }\DecValTok{2}\NormalTok{)}
\NormalTok{y <-}\StringTok{ }\KeywordTok{matrix}\NormalTok{(}\KeywordTok{rep}\NormalTok{(}\DecValTok{10}\NormalTok{, }\DecValTok{4}\NormalTok{), }\DecValTok{2}\NormalTok{, }\DecValTok{2}\NormalTok{)}
\NormalTok{x}
\CommentTok{#>      [,1] [,2]}
\CommentTok{#> [1,]    1    3}
\CommentTok{#> [2,]    2    4}
\NormalTok{y}
\CommentTok{#>      [,1] [,2]}
\CommentTok{#> [1,]   10   10}
\CommentTok{#> [2,]   10   10}
\NormalTok{x }\OperatorTok{*}\StringTok{ }\NormalTok{y  }\CommentTok{# element-wise multiplication}
\CommentTok{#>      [,1] [,2]}
\CommentTok{#> [1,]   10   30}
\CommentTok{#> [2,]   20   40}
\NormalTok{x }\OperatorTok{/}\StringTok{ }\NormalTok{y  }\CommentTok{# element-wise division}
\CommentTok{#>      [,1] [,2]}
\CommentTok{#> [1,]  0.1  0.3}
\CommentTok{#> [2,]  0.2  0.4}
\NormalTok{x }\OperatorTok\StringTok{ }\NormalTok{y  }\CommentTok{# true matrix multiplication}
\CommentTok{#>      [,1] [,2]}
\CommentTok{#> [1,]   40   40}
\CommentTok{#> [2,]   60   60}
\end{Highlighting}
\end{Shaded}

R also recycles arguments.

\begin{Shaded}
\begin{Highlighting}[]
\NormalTok{x <-}\StringTok{ }\DecValTok{1}\OperatorTok{:}\DecValTok{10}
\NormalTok{z <-}\StringTok{ }\NormalTok{x }\OperatorTok{+}\StringTok{ }\NormalTok{.}\DecValTok{1}  \CommentTok{# add .1 to each element}
\NormalTok{z}
\CommentTok{#>  [1]  1.1  2.1  3.1  4.1  5.1  6.1  7.1  8.1  9.1 10.1}
\end{Highlighting}
\end{Shaded}

While you usually either want the same length vector or a length one
vector. You are not limited to just these options.

\begin{Shaded}
\begin{Highlighting}[]
\NormalTok{x <-}\StringTok{ }\DecValTok{1}\OperatorTok{:}\DecValTok{10}
\NormalTok{y <-}\StringTok{ }\NormalTok{x }\OperatorTok{+}\StringTok{ }\KeywordTok{c}\NormalTok{(.}\DecValTok{1}\NormalTok{, .}\DecValTok{2}\NormalTok{) }
\NormalTok{y}
\CommentTok{#>  [1]  1.1  2.2  3.1  4.2  5.1  6.2  7.1  8.2  9.1 10.2}
\NormalTok{z <-}\StringTok{ }\NormalTok{x }\OperatorTok{+}\StringTok{ }\KeywordTok{c}\NormalTok{(.}\DecValTok{1}\NormalTok{, .}\DecValTok{2}\NormalTok{, .}\DecValTok{3}\NormalTok{)}
\CommentTok{#> Warning in x + c(0.1, 0.2, 0.3): longer object length is not a multiple of}
\CommentTok{#> shorter object length}
\NormalTok{z}
\CommentTok{#>  [1]  1.1  2.2  3.3  4.1  5.2  6.3  7.1  8.2  9.3 10.1}
\end{Highlighting}
\end{Shaded}

\subsection{Example}\label{example}

One (not so good) way to estimate \texttt{pi} is through Monte-Carlo
simulation.

Suppose we wish to estimate the value of \texttt{pi} using a Monte-Carlo
method. Essentially, we throw darts at the unit square and count the
number of darts that fall within the unit circle. We'll only deal with
quadrant one. Thus the \(Area = \frac{\pi}{4}\)

Monte-Carlo pseudo code:

\begin{enumerate}
\def\labelenumi{\arabic{enumi}.}
\tightlist
\item
  Initialize \texttt{hits\ =\ 0}
\item
  \textbf{for i in 1:N}
\item
  Generate two random numbers, \(U_1\) and \(U_2\), between 0 and 1
\item
  If \(U_1^2 + U_2^2 < 1\), then \texttt{hits\ =\ hits\ +\ 1}
\item
  \textbf{end for}
\item
  Area estimate = \texttt{hits\ /\ N}
\item
  \(\hat{pi} = 4 * Area Estimate\)
\end{enumerate}

\begin{Shaded}
\begin{Highlighting}[]
\NormalTok{pi_naive <-}\StringTok{ }\ControlFlowTok{function}\NormalTok{(N) \{}
\NormalTok{  hits <-}\StringTok{ }\DecValTok{0}
  \ControlFlowTok{for}\NormalTok{(i }\ControlFlowTok{in} \KeywordTok{seq_len}\NormalTok{(N)) \{}
\NormalTok{    U1 <-}\StringTok{ }\KeywordTok{runif}\NormalTok{(}\DecValTok{1}\NormalTok{)}
\NormalTok{    U2 <-}\StringTok{ }\KeywordTok{runif}\NormalTok{(}\DecValTok{1}\NormalTok{)}
    \ControlFlowTok{if}\NormalTok{ ((U1}\OperatorTok{^}\DecValTok{2} \OperatorTok{+}\StringTok{ }\NormalTok{U2}\OperatorTok{^}\DecValTok{2}\NormalTok{) }\OperatorTok{<}\StringTok{ }\DecValTok{1}\NormalTok{) \{}
\NormalTok{      hits <-}\StringTok{ }\NormalTok{hits }\OperatorTok{+}\StringTok{ }\DecValTok{1}
\NormalTok{    \}}
\NormalTok{  \}}
  
  \DecValTok{4}\OperatorTok{*}\NormalTok{hits}\OperatorTok{/}\NormalTok{N}
\NormalTok{\}}
\NormalTok{N <-}\StringTok{ }\FloatTok{1e6}
\NormalTok{(t1 <-}\StringTok{ }\KeywordTok{system.time}\NormalTok{(}\KeywordTok{pi_naive}\NormalTok{(N)))}
\CommentTok{#>    user  system elapsed }
\CommentTok{#>    4.70    0.04    4.91}
\end{Highlighting}
\end{Shaded}

That's a long run time (and bad estimate). Let's vectorize it.

\begin{Shaded}
\begin{Highlighting}[]
\NormalTok{pi_vect <-}\StringTok{ }\ControlFlowTok{function}\NormalTok{(N) \{}
\NormalTok{  U1 <-}\StringTok{ }\KeywordTok{runif}\NormalTok{(N)}
\NormalTok{  U2 <-}\StringTok{ }\KeywordTok{runif}\NormalTok{(N)}
\NormalTok{  hits <-}\StringTok{ }\KeywordTok{sum}\NormalTok{(U1}\OperatorTok{^}\DecValTok{2} \OperatorTok{+}\StringTok{ }\NormalTok{U2}\OperatorTok{^}\DecValTok{2} \OperatorTok{<}\StringTok{ }\DecValTok{1}\NormalTok{)}
  \DecValTok{4}\OperatorTok{*}\NormalTok{hits}\OperatorTok{/}\NormalTok{N}
\NormalTok{\}}
\NormalTok{(t2 <-}\StringTok{ }\KeywordTok{system.time}\NormalTok{(}\KeywordTok{pi_vect}\NormalTok{(N)))}
\CommentTok{#>    user  system elapsed }
\CommentTok{#>    0.08    0.02    0.09}
\end{Highlighting}
\end{Shaded}

The speed up from vectorization is impressive.

\begin{Shaded}
\begin{Highlighting}[]
\KeywordTok{floor}\NormalTok{(t1}\OperatorTok{/}\NormalTok{t2)}
\CommentTok{#>    user  system elapsed }
\CommentTok{#>      58       2      54}
\end{Highlighting}
\end{Shaded}

\section{Reading For Next Class}\label{reading-for-next-class-1}

\begin{enumerate}
\def\labelenumi{\arabic{enumi}.}
\tightlist
\item
  Read the chapter on \href{http://r4ds.had.co.nz/tibbles.html}{Tibbles}
\end{enumerate}

\section{Exercises}\label{exercises-1}

\begin{enumerate}
\def\labelenumi{\arabic{enumi}.}
\tightlist
\item
  Browse this \href{http://adv-r.had.co.nz/Vocabulary.html}{vocabulary
  list} and read the help file for functions that interest you.
\item
  Re-run the three cases in the For loop section with
  \texttt{x\ \textless{}-\ NULL}
\item
  Vectorization / function practice.
\end{enumerate}

We'll calculate pi using the Gregory-Leibniz series. Mathematicians will
be quick to point out that this is a poor way to calculate pi, since the
series converges very slowly. But our goal is not calculating pi, our
goal is examining the performance benefit that be be achieved using
vectorization.

Here is a formula for the Gregory-Leibniz series:

\begin{equation}
1 - \frac{1}{3} + \frac{1}{5} - \frac{1}{7} + \frac{1}{9} - \frac{1}{11} + \cdots = \frac{\pi}{4}
\end{equation}

Here is the Gregory-Leibniz series in summation notation:

\begin{equation}
\sum_{\text{n}=0}^{\infty} \frac{(-1)^n}{2\cdot n + 1} = \frac{\pi}{4}
\end{equation}

The straightforward implementation using an R loop would look like this:

\begin{Shaded}
\begin{Highlighting}[]
\NormalTok{GL_naive <-}\StringTok{ }\ControlFlowTok{function}\NormalTok{ (limit) \{}
\NormalTok{  p =}\StringTok{ }\DecValTok{0}
  \ControlFlowTok{for}\NormalTok{ (n }\ControlFlowTok{in} \DecValTok{0}\OperatorTok{:}\NormalTok{limit) \{}
\NormalTok{    p =}\StringTok{ }\NormalTok{(}\OperatorTok{-}\DecValTok{1}\NormalTok{)}\OperatorTok{^}\NormalTok{n}\OperatorTok{/}\NormalTok{(}\DecValTok{2} \OperatorTok{*}\StringTok{ }\NormalTok{n }\OperatorTok{+}\StringTok{ }\DecValTok{1}\NormalTok{) }\OperatorTok{+}\StringTok{ }\NormalTok{p}
\NormalTok{    \}}
  \DecValTok{4}\OperatorTok{*}\NormalTok{p}
\NormalTok{\}}

\NormalTok{N <-}\StringTok{ }\FloatTok{1e7}
\KeywordTok{system.time}\NormalTok{(pi_est <-}\StringTok{ }\KeywordTok{GL_naive}\NormalTok{(N))}
\CommentTok{#>    user  system elapsed }
\CommentTok{#>    2.61    0.00    2.67}
\NormalTok{pi_est}
\CommentTok{#> [1] 3.14}
\end{Highlighting}
\end{Shaded}

Your task is to vectorize this function. Do not use any looping or apply
functions. This one is a bit tricky. Hint: It may be easier to think
about it in terms of the series notation and not the summation notation.

\begin{Shaded}
\begin{Highlighting}[]
\NormalTok{GL_vect <-}\StringTok{ }\ControlFlowTok{function}\NormalTok{(limit) \{}
  \CommentTok{# your code here}
  \CommentTok{# use only base functions and no looping mechanisms}
\NormalTok{\}}
\end{Highlighting}
\end{Shaded}

\part{Appendix}\label{part-appendix}

\chapter{Appendix A}\label{appendix-resources}

\section{E-Books}\label{e-books}

\begin{itemize}
\tightlist
\item
  \href{https://bookdown.org/rdpeng/rprogdatascience/}{R Programming for
  Data Science} by Roger D. Peng,
\item
  \href{https://bookdown.org/csgillespie/efficientR/}{Efficient R
  programming} by Colin Gillespie \& Robin Lovelace,
\item
  \href{https://bookdown.org/rdpeng/RProgDA/}{Mastering Software
  Development in R} by Roger D. Peng, Sean Kross, and Brooke Anderson
\item
  \href{https://github.com/lgatto/2016-02-25-adv-programming-EMBL}{Course
  on R debugging and robust programming} by Laurent Gatto \& Robert
  Stojnic,
\item
  \href{https://bookdown.org/rdpeng/RProgDA/}{Mastering Software
  Development in R} by Roger D. Peng, Sean Kross and Brooke Anderson,
\item
  \href{http://r4ds.had.co.nz/index.html}{R for Data Science} by Garrett
  Grolemund \& Hadley Wickham
\item
  \href{http://adv-r.had.co.nz/}{Advanced R} by Hadley Wickham
\item
  \href{http://r-pkgs.had.co.nz/}{R packages} by Hadley Wickham,
\item
  other resources linked from this material.
\end{itemize}

\bibliography{book.bib,packages.bib}


\end{document}
